% !TEX root = system_validation.tex

\subsection{Locomotion}
\label{sec:failure_locomotion}

\subsubsection{Inaccurate Motion}
\label{sec:locomotion_fm_motion}
\added[remark={RH, V2}]{\textbf{Description:} This failure mode describes the situation in which a robot agent is unable to accurately follow motion commands. An example of this is if the robot is commanded to move 10 inches, but localization detects it only moving 4 inches.}\\
\added[remark={RH, V2}]{\textbf{Cause:} Inaccurate motion could be a result of slippage of wheels, failed motor encoders, or failed driving motors. Encoders and the localization system can be used to determin which of these causes occured.}\\
\added[remark={RH, V2}]{\textbf{Effects:} Inability to move accurately can cause a user-reported error, which the user can resolve to continue operation.}\\
\added[remark={RH, V2}]{\textbf{Criticality:}  This is a minor failure, as it does not end system operation and can be resolved by the user.}\\
\added[remark={RH, V2}]{\textbf{Safety Hazards:} The only safety hazard is with regard to the drawing surface; slippage could cause minor destruction of the surface.}\\
\added[remark={DZ, V2}]{\textbf{Mitigation:} Proper maintenance on the driving mechanism and using high-quality, reliable components can help mitigate this failure mode.}


\subsubsection{Failure to Move Omnidirectionally}
\label{sec:locomotion_fm_omni}
\added[remark={RH, V2}]{\textbf{Description:} Failure to make omnidirectional movements means a robot agent cannot move in an arbitrary direction on the flat plane represented by the drawing surface.}\\
\added[remark={RH, V2}]{\textbf{Cause:} Similar to \sref{sec:locomotion_fm_motion}, this could be caused by wheels slippage. Alternatively, a broken wheel or motor could have this effect as well.}\\
\added[remark={RH, V2}]{\textbf{Effects:} Failing to move omnidirectionally could result in incorrect drawings - the robot agent can no longer move along sharp curves to faithfully recreate the input image. When detected, robot operation should halt and robot should report to the user of this failure.}\\
\added[remark={RH, V2}]{\textbf{Criticality:}  This is a critical error, as it has a direct influence on the quality of the drawing.}\\
\added[remark={RH, V2}]{\textbf{Safety Hazards:} As with \sref{sec:locomotion_fm_motion}, the only safety hazard is that a broken wheel could damage the drawing surface.}\\
\added[remark={DZ, V2}]{\textbf{Mitigation:} Proper maintenance on the driving mechanism and using high-quality, reliable components can help mitigate this failure mode.}
