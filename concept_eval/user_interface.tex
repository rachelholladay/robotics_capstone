% !TEX root = main.tex

\section{User Interface}
\label{sec:user_interface}
The user interface provides a unified system for user input. This is the system with which the user specifies the input image and monitors the robots' progress. The system will also display any error messages or anomalies detected by the system, and how the user should address them. There will also be a system-wide kill switch in case of emergencies. \\

\noindent
\textbf{Critical Components:} Screen, input device. \\ 
\noindent
\textbf{Planned Prototype:} \added[remark={RH, V2}]{We plan to sketch our a prototype of our user interface.}

\subsection{Use Cases}
\subsubsection{Display Information}
\textbf{Description:} \added[remark={DZ, V2}]{The system is able to show a human user information pertinent to system operation, including but not limited to the location of the robots, their battery level, amount of task completed, estimated time of completion, and any existing obstructions or anomalies.}

\subsection{Requirements Fulfilled}
\added[remark={RH, V2}]{Our prototype will be designed to allow the user to input a drawing plan (FR11), cancel progress in case of emergency (FR13) and provide an intutive experience (FR14). Additionally this user interface must come complete with documentation (NFR1) and error handling (NFR2) and be reliable (NFR8) and within budget (NFR10). A low priority requirement is also to develop a user interface through a mobile app (NFR7).}
