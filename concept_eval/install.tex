% !TEX root = main.tex

\section{Installation}
\label{sec:install}
Setup for the demo of this system requires a drawing surface and setup of the vision markers under specific conditions. Conditions for setup must be in line with the scope and assumptions made for successful operation of the system. Maintaining scope of the project, as described in \njnote{Reference Requirements section 2.3: Product Scope}, the drawing surface must be placed indoors and in an obstacle-free area. For compliance with \njnote{Requirements spec Assumption 2.4 (flat surface)}, the surface chosen must be a flat and homogenous surface.

Once the drawing surface has been placed indoors on a flat surface, vision markers must be set up and calibrated. As per \njnote{Reference trade study on localization system that selects markers}, localization markers must be placed around the edges of the drawing surface. These markers must then be calibrated by the user, who will input marker locations into the system. The localization markers serve as calibrated base points from which the robot agents will determine their position and orientation.

\rhnote{Rubric: Plan clearly specifies how the exhibit will be integrated and installed and the museum.}

\rhnote{I guess explain how this gets run? For our demo you need a flat area/sheet of paper free of obstacles.} \rhnote{Explain how need to set up vision marker}


