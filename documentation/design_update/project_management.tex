% !TEX root = main.tex

\section{Project Management}
\label{sec:project_management}

\subsection{Work Breakdown Schedule}
\label{sec:wbs}

\dznote{Mostly the pictures with explanations. Then have all the dictionaries. Take picture of each dictionary and insert it in (i think this is the easiest)}

In this section, we present the Work Breakdown Schedule for the project.

\begin{figure}[h!]
\centering
\includegraphics[width=0.98\columnwidth]{wbs_schedule/wbs_1_31_17.png}
\caption{Full WBS for the project}
\label{fig:full-wbs}
\end{figure}
\begin{figure}[h!]
\centering
\includegraphics[width=0.98\columnwidth]{wbs_schedule/wbs_electromechanical.png}
\caption{Electromechanical WBS section}
\label{fig:full-wbs}
\end{figure}
\begin{figure}[h!]
\centering
\includegraphics[width=0.98\columnwidth]{wbs_schedule/wbs_software.png}
\caption{Software WBS section}
\label{fig:full-wbs}
\end{figure}
\begin{figure}[h!]
\centering
\includegraphics[width=0.98\columnwidth]{wbs_schedule/wbs_integration.png}
\caption{Integration WBS section}
\label{fig:full-wbs}
\end{figure}

The following WBS dictionary entries include more information on each of the work elements of the project. 

\begin{figure}[h!]
\centering
\includegraphics[width=0.49\columnwidth]{wbs_schedule/wbs_dict_hw1.png}
\includegraphics[width=0.49\columnwidth]{wbs_schedule/wbs_dict_hw2.png}
\label{fig:hw1hw2}
\end{figure}
\begin{figure}[h!]
\centering
\includegraphics[width=0.49\columnwidth]{wbs_schedule/wbs_dict_hw3.png}
\includegraphics[width=0.49\columnwidth]{wbs_schedule/wbs_dict_hw4.png}
\label{fig:hw3hw4}
\end{figure}
\begin{figure}[h!]
\centering
\includegraphics[width=0.49\columnwidth]{wbs_schedule/wbs_dict_hw5.png}
\includegraphics[width=0.49\columnwidth]{wbs_schedule/wbs_dict_hw6.png}
\label{fig:hw5hw6}
\end{figure}
\begin{figure}[h!]
\centering
\includegraphics[width=0.49\columnwidth]{wbs_schedule/wbs_dict_hw7.png}
\includegraphics[width=0.49\columnwidth]{wbs_schedule/wbs_dict_sw1.png}
\label{fig:hw7sw1}
\end{figure}
\begin{figure}[h!]
\centering
\includegraphics[width=0.49\columnwidth]{wbs_schedule/wbs_dict_sw2.png}
\includegraphics[width=0.49\columnwidth]{wbs_schedule/wbs_dict_sw3.png}
\label{fig:sw2sw3}
\end{figure}
\begin{figure}[h!]
\centering
\includegraphics[width=0.49\columnwidth]{wbs_schedule/wbs_dict_sw4.png}
\includegraphics[width=0.49\columnwidth]{wbs_schedule/wbs_dict_sw5.png}
\label{fig:sw4sw5}
\end{figure}
\begin{figure}[h!]
\centering
\includegraphics[width=0.49\columnwidth]{wbs_schedule/wbs_dict_sw6.png}
\includegraphics[width=0.49\columnwidth]{wbs_schedule/wbs_dict_sw7.png}
\label{fig:sw6sw7}
\end{figure}
\begin{figure}[h!]
\centering
\includegraphics[width=0.49\columnwidth]{wbs_schedule/wbs_dict_int1.png}
\includegraphics[width=0.49\columnwidth]{wbs_schedule/wbs_dict_int2.png}
\label{fig:int1int2}
\end{figure}
\begin{figure}[h!]
\centering
\includegraphics[width=0.49\columnwidth]{wbs_schedule/wbs_dict_int3.png}
\label{fig:int3}
\end{figure}

\subsection{Schedule}
\label{sec:schedule}
Scheduling for the semester has been split into three main sections, as outlined in the wbs (\sref{sec:wbs}). These sections were determined into electromechanical, software, and integration. Both electromechanical and software development can be implemented and built simultaneously, with integration following once both pieces are complete. By developing hardware and software at the same time, the team can make adjustments to both systems based on changes to the other. We planned the schedule to allow the last month for integration and testing, which will help us to ensure the full system works for the final demo. \\
We chose to represent the schedule as a Google Calendar, which allows us to integrate it with our schedules for other classes, as well as giving us convenient access and use. This can also be represented as a Gantt Chart, as in \figref{fig:gantt_chart}.

\begin{figure}[ht!]
 \centering
  \includegraphics[width=0.99\columnwidth]{figures/gantt_chart.png}
  \caption{Gantt Chart of the semester schedule}
 \label{fig:gantt_chart}
\end{figure}
