% !TEX root = main.tex

\section{Use Cases}
\label{sec:use_cases}
%Rubric: Use cases provide clear insight into the system and cover a wide range of possible interactions. Develop complete use case and detailed scenarios for the concept

\begin{figure}
 \centering
  \includegraphics[width=0.6\columnwidth]{diagrams/use-cases-diagram.jpg}
	\caption{Diagram representing the relationship between use cases and actors in the system}
 \label{fig:use-cases}
\end{figure}

\subsection{Load/Switch Writing Implement}
\textbf{Summary:} A robot's main consumable, the writing implement, must be reloaded or replaced when empty or when a different implement (with different color stroke, or other properties) is desired. \\
\textbf{Actors:}  Writing implement, human \\
\textbf{Precondition:} No writing tool in the system or the one that is currently loaded is no longer needed for the task. \\
\textbf{Post-condition:} System has desired writing tool loaded. \\
\textbf{Alternative:} If the robot is unable to load the implement then the system should report improper loading results to the user. Additionally, other robot agents should be informed that the robot to be reloaded is broken, and therefore cannot draw. Hence, in order to complete the picture the other robots must re-plan accordingly. \\
\textbf{Description:} When the robot is working on a large or intricate drawing that requires a large amount of consumable writing material, the robot may not be able to carry enough material to complete its allocation of the drawing. Additionally, some drawings may involve a variety of colors, strokes, or other properties that necessitate the use of different writing materials. Thus, the robot must be able to replace its writing implement with human assistance. By using a modular writing implement actuator, the robot can use a variety of different writing utensils - for example, by switching out the actuator the robot can go from being equipped to use chalk to being equipped to use permanent marker. The user can simply pull the previous writing implement out of the actuator, and insert the new implement, which is held in place by friction. The robot must be able to recognize reloading failures and alert the human operator of their occurrences through the user interface (\sref{sec:subsystem-user-interface}). If the failure is not corrected, The robot must communicinpate its inability to perform to other robots in the system so they can re-plan drawing paths accordingly. For further details, see the Writing Implement Actuator subsystem description (\sref{sec:subsystem_writing_implement_actuator}).\\

\subsection{Process Input Image}
\textbf{Summary:} The system must take in a human-provided image and interpret what the robot system is required to draw. \\
\textbf{Actors:} Image, Human \\
\textbf{Precondition:}  No existing image being actively drawn by the robot system. \\
\textbf{Post-condition:} The image is processed and ready for work distribution between robots. \\
\textbf{Alternative:} If this step fails the robotic system should report an image processing failure. The user must then address the error before the system can continue work.\\
\textbf{Description:} The system's task will be input using a human-produced image following specific guidelines. From the image, the system can determine where to place markings in the real world, and how the work should be distributed amongst the robot workers. Parameters such as the scale and orientation of the final drawing are also specified at this stage. For further details, see the Input Image Scanner and Image Processor subsystem descriptions (\sref{sec:subsystem_input_scanner} and \sref{sec:subsystem_image_processor} respectively).\\

\subsection{Localization}
\textbf{Summary:} The robot must determine where it is in the working area. \\
\textbf{Actors:} Environment \\
\textbf{Precondition:} The robot is uncertain as to its position and orientation. \\
\textbf{Post-condition:} Following localization we hope to have minimal uncertainty about the current location and orientation of the robot. \\
\textbf{Alternative:} If the robot cannot localize then this should be reported to human user and the other robots should be alerted of the robot's inability to localize and orient itself. Additionally, any robot that cannot localize can potentially draw incorrectly or collide with another component of the system. Thus any robot that fails to localize should halt all movement until localization capability is reestablished.\\
\textbf{Description:} In order to create an accurate reproduction of the input image, the robot needs to know how its location maps to a location on the input image. The system will use a beacon/marker system to determine a robot's position and orientation. If it is unable to do so with a particular robot, that robot cannot continue drawing and the system must alert other robots to the fact so that they can re-plan and reschedule the workload. For further details, see the Localization subsystem description (\sref{sec:subsystem_localization}).\\

\subsection{Scheduling and Robot Planning/Coordination}
\textbf{Summary:} The robot workers must determine an efficient allocation of the work. \\
\textbf{Actors:} None \\
\textbf{Precondition:} No plan or schedule currently exists. \\
\textbf{Post-condition:} Each robot has an allocation of work and a planned path. \\
\textbf{Alternative:} If planning fails, the human user is alerted and the operation is aborted. The user must address the error before operation can continue.\\
\textbf{Description:} The work required must be determined from the input image, and analyzed to determine an efficient allocation of work, as well as a path schedule for each of the robots. Additionally, robots must communicate work completed and failure states to each other in case re-planning is required. For further details, see the Scheduler/Planner subsystem description (\sref{sec:subsystem_planner}).\\

\subsection{Move Robot}
\textbf{Summary:} The robot moves across flat terrain. \\
\textbf{Actors:} Ground \\
\textbf{Precondition:} Robot is stationary. \\
\textbf{Post-condition:} Robot is moving as commanded. \\
\textbf{Alternative:} If the robot is unable to move, the the human user is alerted and work is redistributed between remaining robots. \\
\textbf{Summary:} The robot moves across the writing surface. By using a holonomic drive mechanism, the robot can make sharp turns and intricate designs. For further details, see the Locomotion subsystem description (\sref{sec:subsystem_locomotion}).\\

\subsection{Use Writing Implement}
\textbf{Summary:} The robot creates a mark on the writing surface. \\
\textbf{Actors:} Writing surface, Writing implement \\
\textbf{Precondition:} The robot has a plan to draw an mark, but that mark has not currently be drawn yet. \\
\textbf{Post-condition:} The robot's planned mark is drawn on the writing surface. \\
\textbf{Alternative:} If the robot cannot write, the robot alerts user that the writing implement must be replaced (or inserted if one does not exist). \\
\textbf{Summary:} The robot creates markings on the surface with the given writing implement, and ensures that its movements do not disrupt drawing accuracy. For further details, see the Writing Implement Actuator subsystem description (\sref{sec:subsystem_writing_implement_actuator}).\\
