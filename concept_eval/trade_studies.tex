% !TEX root = main.tex

\section{Trade Studies}
\label{sec:trade_studies}

In creating our multi-agent drawing robot, we face design trade-off and decisions for many of our major subsystems detailed in \sref{sec:subsystem_descriptions}. We investigate four major decision points in the following brief trade studies. Our discussions led to a few technical solutions, when we feel we have enough data to make a reasonable decision. However, for some of our discussion points, further testing and prototyping is needed to evaluate our options fairly. For these points we lay out brief prototyping plans below. 

\subsection{Localization Method}
\label{sec:trade_localization}
Localization is an important factor for multi-agent planning that must be accurate in both position and orientation. Keeping in mind limitations in price and ease of use, there are two major methods for localization: vision-based, and marker based. They are described below.

Vision-based localization involves using cameras or other sensors to directly obtain information on the environment, and then localize the robots in the system based on found landmarks in that environment. One example of this is SLAM (Simultaneous Localization and Mapping), often used by autonomous vehicles to build a map and localize to it at the same time. In this system, robots could build a small map of their surroundings, then match their location to features they find in the environment. Benefits include being location agnostic, and requiring no additional parts or external setup. However, pure vision systems are difficult to calibrate and accuracy can depend heavily on static surroundings, which is not something this system can guarantee.

The other main choice of methodology is marker-based. Using markers placed around the drawing surface, robot agents can quickly locate these markers and their position relative to each one, and consequently triangulate their position and orientation. While requiring additional setup and more parts to calibrate than vision-based localization, existing technology makes it convenient and cheap to get marker-based localization working quickly. One example of a marker-based localization system is AprilTags \cite{olson2011apriltag}, which can be described as 2D barcodes placed in a scene. Within marker localization, however, are two subcategories: passive and active. Passive markers do not output any information, and exist for the robot agent to observe and triangulate accordingly. AprilTags are an example of a passive marker system. On the other hand, active markers will \"look at\" robot agents to determine where the agents are, rather than the robots searching for markers. While less common, active systems behave well in conditions when the markers may not always be easily visible to robot agents \cite{cassinis2005active}.

Given the convenience and ease of use of marker-based localization, it is clearly the better choice. However, it is more difficult to determine whether using active or passive systems will be more effective. A prototype for each system should be made for further evaluation.

\subsection{Locomotion Method}
\label{sec:trade_locomotion}
One of our functional requirement with the highest priority (FR1) is to be able to move in specified directions with a high degree of accuracy.  Additionally, a medium high priority functional requirement (FR 9) is a drive control system that enables this movement. 

Therefore, it is clear from our requirements that locomotion is critical to our robot's performance. When considering locomotion options there are three large categories: flight, legged and wheeled. 

Our drawing occurs on the two dimensional plane. Therefore, any flight based system would have to heavily constrain its third dimensional of position and two additional dimensions of orientation when drawing. Practically speaking, this seems to over-complicate the problem with little additional gain. 

Therefore we are narrowed to a legged or wheeled system.  We are inspired by the wide spread success of wheeled robotic systems and especially appreciate their stability, a critical concern when carefully drawing images. While a legged system would have the ability to traverse uneven terrain, this is not necessary in our project due to our assumption (A3) that the drawing surface is flat and homogenous. 

Therefore, in our analysis we have concluded that our agent's drive system should be wheeled. In our next analysis we further detail the type of drive train. 

\subsection{Drive System}
\label{sec:trade_drive}
In continuing to prioritize our functional requirement (FR1) of being able to in specified directions, we investigate possible drive systems. 

Three similar drive systems first come to mind: differential drive, ackerman steering and four-wheel steering. Differential drive on a robot typically consists of two independently driven wheels and a non-driven wheel. Moving up in complexity and taking after some cars is ackerman steering, where the back pair of wheels is fixed in orientation but the front pair can pivot. In more complex is four-wheel steering, which is similar to ackerman steering but allows the back wheels to also pivot. However, these systems suffer from the fundamental issue of being nonholonomic. Therefore, while they could achieve full mobility, this might complicate drawing and path planning. 

Therefore instead we want an omnidirectional base that is holonomic. This can be achieved with omniwheels, perhaps more commonly, with mecanum wheels. 

\subsection{Drawing Tool}
\rhnote{No sure. Need to prototype. }

%Rubric: Conduct trade studies when there are significant alternatives. Describe trade studies (use comparative analysis techniques) and results. Trade studies present for all major design decisions and clearly identify reasons for choices. If no choice is made, a prototyping plan is identified.
%According to google: "A trade study or trade-off study is the activity of a multidisciplinary team to identify the most balanced technical solutions among a set of proposed viable solutions (FAA 2006). These viable solutions are judged by their satisfaction of a series of measures or cost functions."