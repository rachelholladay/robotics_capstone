% !TEX root = main.tex

\section{Concept Operation}
\label{sec:concept_operation}
%Rubric: "Concept of operation clearly and succinctly specifies how the exhibit will work.". Describe the system from the viewpoint of someone using it. 

\njnote{Properly insert user operation diagrams\/concept\_operation\_user.jpg}

A description of the user operation figure above is as follows. For a further explanation of the user interface, see \sref{sec:subsystem_ui}.

Initial setup contains two simultaneous operations: adding the image to be drawn to the system, and setup of the drawing surface. Adding the input image involves generating the image, and then scanning it into the system. This satisfies requirements for input the drawing plan (Requirements Specification 5.1, FR9), and a user interface (Requirements Specification 5.1, FR12). Setup of the system involves placing the drawing surface, placing and calibrating vision markers, and finally placing the robot agents within the bounds of the drawing surface. Once both steps are done, the user can enter any required settings for their use, and begin operation. Processing of the input image is done automatically by the system and is invisible to the user (\sref{sec:subsystem_image_processor}).

Once the autonmous process begins, the user does nothing but observe until drawing completion. However, any errors will be reported to the user, who then has the option of fixing the issue to continue operation, or terminating the drawing process. The autonomous process satisfies functional requirement FR2 (Requirements Specification 5.1, FR2).
