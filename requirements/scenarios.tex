% !TEX root = main.tex

\section{Scenarios}
\label{sec:scenarios}
Include pictures in this section

\subsection{Chalk Drawing}
Chalk drawings are often used around campuses and different communities for aesthetic purpose, information sharing, and events announcements. However, these drawings are often limited by size and complexity of the drawing. Therefore, we plan to design our robotic system to draw large scale items on blacktop or asphalt surfaces with chalks. Drawing designs will be loaded to the control hub which will then analyze the images and generate pathes for each robot to follow. Each individual robot will recieve commends from the control hub and proceed to complete its responsible section. If these robots encounter any errors or difficulties during the painting process, they need to notify the control hub to make further decisions. Since chalk is the main painting tool in this scenario and these robots may work outdoor, we need to design the robot to be ready for drawing on relative wet surfaces and be able to protect chalk from rain or excessive humidity. Chalks' length goes down as being used; a mechanism that is insensible to chalk's length and can ensure chalk tip always be in contact with the ground is needed. 

\subsection{Infrastructure Lanes Drawing}
Drawing infrastructure lanes, like those for parking lots, highways, streets, and airports, can often be dull and expensive; this task can be easily solved by this robotic system. Digital or physical engineering drawings of the intended lanes will be scanned and loaded into the control hub which then delegates tasks to different robots. Each robot will be in charge of drawing lanes within a bounded area defined by the control hub. When drawing lanes, each robot has to travel at a constant speed and in a straight line, because the drawn lanes need to be stright and smooth. Therefore, motor control becomes critical during this process. During the drawing process, if any robot runs out of ink or experiences any sort of accident, it needs to report its status to control hub, which will commend it to return home base to change painting tool, or call other robot to cover its task, or call for human control. When drawing is done, robots will return to home base automatically. Since completing this scenario requires working outdoor, these robots need to be more robust both in hardware and in software. The mechanical parts need to be weather resistant to a certain extent and the programming scripts need to account for possible situations when working outdoor. 

\subsection{Sport Lines}
Drawing lanes for sports fields, including soccer fields, football fields, basketball court, etc, could be another scenario for this robotic system. Similar to the infrastructure lanes drawing, engineering drawings of the designed lanes will be loaded into the control hub which will assign tasks to each individual robot. These robots will then proceed to painting the lanes with the goals of making lanes smooth and straight. If an error occurs with any robot, such error message will be reported to the control hub. Control hub will then makes decision on calling the robot back or calling for human control. Robots will return to home base automatically when finish painting. When painting certain sports fields, like soccer fields or football fields, these robots need to travel on uneven surfaces, like on grass fields. Therefore, better suspension and drive system will be designed to complete these tasks.
