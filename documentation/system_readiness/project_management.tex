% !TEX root = main.tex
\section{Project Management}
\label{sec:project_management}

\subsection{Work Breakdown Schedule}
\label{sec:wbs}

In this section, we present the Work Breakdown Schedule for the project.

\begin{figure}[!ht]
\centering
\includegraphics[width=\textwidth]{wbs_schedule/wbs_1_31_17.png}
\caption{Full WBS for the project}
\label{fig:full-wbs}
\end{figure}

\begin{figure}[!ht]
\centering
\includegraphics[width=0.8\columnwidth]{wbs_schedule/wbs_electromechanical.png}
\caption{Electromechanical WBS section}
\label{fig:full-wbs}
\end{figure}
\begin{figure}[!ht]
\centering
\includegraphics[width=0.9\columnwidth]{wbs_schedule/wbs_software.png}
\caption{Software WBS section}
\label{fig:full-wbs}
\end{figure}
\begin{figure}[!ht]
\centering
\includegraphics[width=0.9\columnwidth]{wbs_schedule/wbs_integration.png}
\caption{Integration WBS section}
\label{fig:full-wbs}
\end{figure}

\clearpage

The WBS dictionary entries include more information on each of the work elements of the project. Information such as estimates for the amount of time each task will take and their dependencies will help us adhere to our schedule, while determining the owner of each task will improve tractability of the workflow.

\begin{figure}[!ht]
\centering
\includegraphics[width=0.98\columnwidth]{wbs_schedule/wbs_dict_hw1.PNG}
\label{fig:hw1}
\end{figure}
\begin{figure}[!ht]
\centering
\includegraphics[width=0.98\columnwidth]{wbs_schedule/wbs_dict_hw2.PNG}
\label{fig:hw2}
\end{figure}
\begin{figure}[!ht]
\centering
\includegraphics[width=0.98\columnwidth]{wbs_schedule/wbs_dict_hw3.PNG}
\label{fig:hw3}
\end{figure}
\begin{figure}[!ht]
\centering
\includegraphics[width=0.98\columnwidth]{wbs_schedule/wbs_dict_hw4.PNG}
\label{fig:hw4}
\end{figure}
\begin{figure}[!ht]
\centering
\includegraphics[width=0.98\columnwidth]{wbs_schedule/wbs_dict_hw5.PNG}
\label{fig:hw5}
\end{figure}
\begin{figure}[!ht]
\centering
\includegraphics[width=0.98\columnwidth]{wbs_schedule/wbs_dict_hw6.PNG}
\label{fig:hw6}
\end{figure}
\begin{figure}[!ht]
\centering
\includegraphics[width=0.98\columnwidth]{wbs_schedule/wbs_dict_hw7.PNG}
\label{fig:hw7}
\end{figure}
\begin{figure}[!ht]
\centering
\includegraphics[width=0.98\columnwidth]{wbs_schedule/wbs_dict_sw1.PNG}
\label{fig:hw1}
\end{figure}
\begin{figure}[!ht]
\centering
\includegraphics[width=0.98\columnwidth]{wbs_schedule/wbs_dict_sw2.PNG}
\label{fig:hw2}
\end{figure}
\begin{figure}[!ht]
\centering
\includegraphics[width=0.98\columnwidth]{wbs_schedule/wbs_dict_sw3.PNG}
\label{fig:hw3}
\end{figure}
\begin{figure}[!ht]
\centering
\includegraphics[width=0.98\columnwidth]{wbs_schedule/wbs_dict_sw4.PNG}
\label{fig:hw4}
\end{figure}
\begin{figure}[!ht]
\centering
\includegraphics[width=0.98\columnwidth]{wbs_schedule/wbs_dict_sw5.PNG}
\label{fig:hw5}
\end{figure}
\begin{figure}[!ht]
\centering
\includegraphics[width=0.98\columnwidth]{wbs_schedule/wbs_dict_sw6.PNG}
\label{fig:hw6}
\end{figure}
\begin{figure}[!ht]
\centering
\includegraphics[width=0.98\columnwidth]{wbs_schedule/wbs_dict_sw7.PNG}
\label{fig:hw7}
\end{figure}
\begin{figure}[!ht]
\centering
\includegraphics[width=0.98\columnwidth]{wbs_schedule/wbs_dict_int1.PNG}
\label{fig:int1}
\end{figure}
\begin{figure}[!ht]
\centering
\includegraphics[width=0.98\columnwidth]{wbs_schedule/wbs_dict_int2.PNG}
\label{fig:int1}
\end{figure}
\begin{figure}[!ht]
\centering
\includegraphics[width=0.98\columnwidth]{wbs_schedule/wbs_dict_int3.PNG}
\label{fig:int1}
\end{figure}

\clearpage

\subsection{Schedule}
\label{sec:schedule}

\figref{fig:gantt_4_3} shows our current progress towrads meeting our schedule. At this point, the electromechanical design is slightly behind - the camera rig assembly is still being built, and the second robot has not been constructed yet. The camera rig has been recently updated in design, and parts have been ordered. We do not expect assembly to take long, or hinder progress in integration. The second robot parts have already been ordered. We delayed building it to ensure we could finalize the robot design before building a second one.

Software implementation at this point is complete. All of the major subsystems are functional, and the only additions are to enhance or add additional features. Other software changes are being done for integration, to allow the various subsystems to work with the hardware, or with each other. The software design and implementation has reached usability and is therefore on schedule.

Integration is will underway, which is the majority of the work we have left. Some subsystems have been integrated, and others are actively being worked on. Motion onboard the robot is completed, and there are plans to finish integrating the communication and localization subsystems within the next week.

\begin{figure}[!ht]
\centering
\includegraphics[width=0.6\columnwidth]{figs/gantt_chart_4_3_17.png}
\caption{Schedule via Gantt Chart}
\label{fig:gantt_4_3}
\end{figure}
