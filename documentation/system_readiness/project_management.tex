% !TEX root = main.tex
\section{Project Management}
\label{sec:project_management}

\subsection{Work Breakdown Schedule}
\label{sec:wbs}

In this section, we present the Work Breakdown Schedule for the project.

\begin{figure}[h!]
\centering
\includegraphics[width=\textwidth]{wbs_schedule/wbs_1_31_17.png}
\caption{Full WBS for the project}
\label{fig:full-wbs}
\end{figure}

\begin{figure}[h!]
\centering
\includegraphics[width=0.8\columnwidth]{wbs_schedule/wbs_electromechanical.png}
\caption{Electromechanical WBS section}
\label{fig:full-wbs}
\end{figure}
\begin{figure}[h!]
\centering
\includegraphics[width=0.9\columnwidth]{wbs_schedule/wbs_software.png}
\caption{Software WBS section}
\label{fig:full-wbs}
\end{figure}
\begin{figure}[h!]
\centering
\includegraphics[width=0.9\columnwidth]{wbs_schedule/wbs_integration.png}
\caption{Integration WBS section}
\label{fig:full-wbs}
\end{figure}

\clearpage

The WBS dictionary entries include more information on each of the work elements of the project. Information such as estimates for the amount of time each task will take and their dependencies will help us adhere to our schedule, while determining the owner of each task will improve tractability of the workflow.

\begin{figure}[h!]
\centering
\includegraphics[width=0.98\columnwidth]{wbs_schedule/wbs_dict_hw1.PNG}
\label{fig:hw1}
\end{figure}
\begin{figure}[h!]
\centering
\includegraphics[width=0.98\columnwidth]{wbs_schedule/wbs_dict_hw2.PNG}
\label{fig:hw2}
\end{figure}
\begin{figure}[h!]
\centering
\includegraphics[width=0.98\columnwidth]{wbs_schedule/wbs_dict_hw3.PNG}
\label{fig:hw3}
\end{figure}
\begin{figure}[h!]
\centering
\includegraphics[width=0.98\columnwidth]{wbs_schedule/wbs_dict_hw4.PNG}
\label{fig:hw4}
\end{figure}
\begin{figure}[h!]
\centering
\includegraphics[width=0.98\columnwidth]{wbs_schedule/wbs_dict_hw5.PNG}
\label{fig:hw5}
\end{figure}
\begin{figure}[h!]
\centering
\includegraphics[width=0.98\columnwidth]{wbs_schedule/wbs_dict_hw6.PNG}
\label{fig:hw6}
\end{figure}
\begin{figure}[h!]
\centering
\includegraphics[width=0.98\columnwidth]{wbs_schedule/wbs_dict_hw7.PNG}
\label{fig:hw7}
\end{figure}
\begin{figure}[h!]
\centering
\includegraphics[width=0.98\columnwidth]{wbs_schedule/wbs_dict_sw1.PNG}
\label{fig:hw1}
\end{figure}
\begin{figure}[h!]
\centering
\includegraphics[width=0.98\columnwidth]{wbs_schedule/wbs_dict_sw2.PNG}
\label{fig:hw2}
\end{figure}
\begin{figure}[h!]
\centering
\includegraphics[width=0.98\columnwidth]{wbs_schedule/wbs_dict_sw3.PNG}
\label{fig:hw3}
\end{figure}
\begin{figure}[h!]
\centering
\includegraphics[width=0.98\columnwidth]{wbs_schedule/wbs_dict_sw4.PNG}
\label{fig:hw4}
\end{figure}
\begin{figure}[h!]
\centering
\includegraphics[width=0.98\columnwidth]{wbs_schedule/wbs_dict_sw5.PNG}
\label{fig:hw5}
\end{figure}
\begin{figure}[h!]
\centering
\includegraphics[width=0.98\columnwidth]{wbs_schedule/wbs_dict_sw6.PNG}
\label{fig:hw6}
\end{figure}
\begin{figure}[h!]
\centering
\includegraphics[width=0.98\columnwidth]{wbs_schedule/wbs_dict_sw7.PNG}
\label{fig:hw7}
\end{figure}
\begin{figure}[h!]
\centering
\includegraphics[width=0.98\columnwidth]{wbs_schedule/wbs_dict_int1.PNG}
\label{fig:int1}
\end{figure}
\begin{figure}[h!]
\centering
\includegraphics[width=0.98\columnwidth]{wbs_schedule/wbs_dict_int2.PNG}
\label{fig:int1}
\end{figure}
\begin{figure}[h!]
\centering
\includegraphics[width=0.98\columnwidth]{wbs_schedule/wbs_dict_int3.PNG}
\label{fig:int1}
\end{figure}

\clearpage

\subsection{Objectives Tree}
\label{sec:objectives_tree}
We created an objectives tree to better organize our requirements, and prioritize them based on system purposes and goals. After analysis of our requirements, we
formed the objectives tree in \figref{fig:obj_tree_full} with the following categories:

\begin{enumerate}
\item Is Safe (\figref{fig:obj_tree_safe})
\item Is Portable (\figref{fig:obj_tree_portable})
\item Drawing Tool is Easy to Operate (\figref{fig:obj_tree_tool})
\item Is Mobile (\figref{fig:obj_tree_mobile})
\item User-Friendly (\figref{fig:obj_tree_user})
\item Performance Guarantees (\figref{fig:obj_tree_performance})
\end{enumerate}

The category for `Is Safe' (\figref{fig:obj_tree_safe}) encompasses requirements for the robot staying within bounds, maintaining reliable communication, existence of a kill switch, and overall safe operation. These requirements breakdown how safe usage of the robot can be achieved, through both system design and user operation.

`Is Portable' (\figref{fig:obj_tree_portable}) specifies system constraints that enable the robots to be able to be transported easily. The battery-powered requirement ensures the robots do not need external power during operation. Weight and size requirements were further categorized into physical constraints, to emphasize the importance of those requirements on portability outside of system operation.

Subtree `Drawing Tool is Easy to Operate' (\figref{fig:obj_tree_tool}) ensures the writing tool is easy to maintain and use both during and before or after system operation. The main subtree describes tool maintenance. Requirements under maintenance include inserting, removing, and replacing the tool, as well as duration requirements for replacing the writing tool. Other requirements in this category relate to having the ability to enage or disengage the writing tool. This requirement is involved with system operation, and ensures the robot can change the tool status so the robots can move regardless of whether it is drawing.

The `Is Mobile' (\figref{fig:obj_tree_mobile}) tree categorizes mobility requirements and constraints for the robot agents. Both positional and rotational accuracy are categorized under their own Accuracy subtree. Other leaves in this subtree ensure the robot agents have their own drive control systems, can localize, and are able to move autonomously in any direction on a 2D plane.

We also chose to separate out requirements that relate to engaging the user and  enable a user-friendly experience. These fall under the `User-Friendly' subtree (\figref{fig:obj_tree_user}). Both documentation and budget requirements were categorized here - these requirements are more likely to be for users interested in adapting or recreating our system. As a result, other requirements were further categorized into a user-interaction subtree. These constraints denote existence of a UI, error handling, and the ability for users to input their drawing plan.

The final categorization, `Performance Guarantees' (\figref{fig:obj_tree_performance}) denotes overall system requirements to ensure the final drawing meets specifications. These requirements include ensuring the robots know their own progress, and coordinate with each other. In addition, requirements specifying system efficiency, reliability, and overall drawing quality fell into this category.

\begin{figure}[h!]
\centering
\includegraphics[width=0.98\columnwidth]{figs/objectives_tree/objectives_tree_4.2.17.png}
\label{fig:obj_tree_full}
\end{figure}
\begin{figure}[h!]
\centering
\includegraphics[width=0.98\columnwidth]{figs/objectives_tree/objectives_tree_safe.png}
\label{fig:obj_tree_safe}
\end{figure}
\begin{figure}[h!]
\centering
\includegraphics[width=0.98\columnwidth]{figs/objectives_tree/objectives_tree_portable.png}
\label{fig:obj_tree_portable}
\end{figure}
\begin{figure}[h!]
\centering
\includegraphics[width=0.98\columnwidth]{figs/objectives_tree/objectives_tree_easy_tool_ops.png}
\label{fig:obj_tree_tool}
\end{figure}
\begin{figure}[h!]
\centering
\includegraphics[width=0.98\columnwidth]{figs/objectives_tree/objectives_tree_mobile.png}
\label{fig:obj_tree_mobile}
\end{figure}
\begin{figure}[h!]
\centering
\includegraphics[width=0.98\columnwidth]{figs/objectives_tree/objectives_tree_user_friendly.png}
\label{fig:obj_tree_user}
\end{figure}
\begin{figure}[h!]
\centering
\includegraphics[width=0.98\columnwidth]{figs/objectives_tree/objectives_tree_performance.png}
\label{fig:obj_tree_performance}
\end{figure}

\subsection{Schedule}
\label{sec:schedule}

\figref{fig:gantt_4_3} shows our current progress towrads meeting our schedule. At this point, the electromechanical design is slightly behind - the camera rig assembly is still being built, and the second robot has not been constructed yet. The camera rig has been recently updated in design, and parts have been ordered. We do not expect assembly to take long, or hinder progress in integration. The second robot parts have already been ordered. We delayed building it to ensure we could finalize the robot design before building a second one.

Software implementation at this point is complete. All of the major subsystems are functional, and the only additions are to enhance or add additional features. Other software changes are being done for integration, to allow the various subsystems to work with the hardware, or with each other. The software design and implementation has reached usability and is therefore on schedule.

Integration is will underway, which is the majority of the work we have left. Some subsystems have been integrated, and others are actively being worked on. Motion onboard the robot is completed, and there are plans to finish integrating the communication and localization subsystems within the next week.

\begin{figure}[h!]
\centering
\includegraphics[width=0.6\columnwidth]{figs/gantt_chart_4_3_17.png}
\caption{Schedule via Gantt Chart}
\label{fig:gantt_4_3}
\end{figure}
