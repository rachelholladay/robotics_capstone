% !TEX root = main.tex
\section{Future Work and Commercialization Considerations}
\label{sec:future_work}
We next look toward the future on what factors we would need to consider to develop our capstone into a full product and company. 

\subsection{Economic Factors}
Having a sustainable business model is imperative to successful commercialization of our drawing agent system. The main economic considerations relate to selling the robots and accessories, production, and the main product to be used for profits. We would be looking at producing or selling paint and the holders, spare parts, and the robots themselves. Each of these must then take into account the need to produce them at scale for mass production. We could also consider selling robots of various sizes, for varying purposes.This could include airport runway lines, sports field lines, or more consumer-focused as a toy or tool for advertising on the ground. Our business models could include usage training, renting vs. buying individual units, and maintenance costs.

\subsection{Reliability}
To ensure our robot system is reliable for commercialization, we have considered a number of changes that would need to be made. Primarily, the system would need to be more robust - this includes the ability to work on surfaces that are not completely flat, robot weatherproofing, and updating durability to use something stronger than wood. Usage changes would allow the system to communicate consistently without latency, and take advantage of motor encoders or other sensors to significantly improve drawing accuracy. Finally, we would need a consistent and standardized methodology to verify the quality of finished drawings.

\subsection{Safety}
While the robots built for this project are small, updates in size or usability could potentially pose safety risks that must be accounted for. That in mind, a major change we would make to improve safety would be with regard to collision detection. Robot agents should be able to consistently avoid collisions with each other and react accordingly, as well as detect and halt operation in case of near-collision with other obstacles (such as bystanders). The writing tool would also need a more straightforward method for loading. Currently it involves reaching inside the robot, which could pose dangerous for inexperienced or young users. Finally, any use case in a public space would require coordination with local authorities. For airports, this would be air traffic control, for road painting, coordinating with the local transportation departments to work out road closings or detours.

\subsection{Maintainability}
Commercialization would also require the robot system to be more easily maintainable, as mass production makes it difficult for a small company or group to repair each robot if they constantly break. Simpler updates would be longer battery life, better resource management (this includes both drawing material and battery), and a more robust painting mechanism with easier assembly and setup. Aside from physical changes, maintenance plans could include service plans, geolocation as a feature for users, and straightforward documentation to assist users in system operation.

\subsection{Sustainability}
Sustainability is a common marketing tool, and this robot system lends itself fairly simply to being a more sustainable product. The drawing material could be updated to use an eco-friendly paint, and the materials and manufacturing process could easily be updated for environmental friendliness. Other considerations could allow the robot agents to use solar power, or have a charging system to avoid use of standard single-use batteries. Finally, systems could be provided to easily recycle the robots when and if they eventually do stop working.

\subsection{Ethical Issues}
Building a business around these robots comes with a couple of ethical issues, similar to any new product. Liability for accidents is a concern that would have to be dealt with, as well as constructing or designing theft-deterring mechanisms to help users feel more comfortable letting the robots run autonomously without supervision.

\subsection{Marketability}
Successful marketing can often make or break a new company. To successfully market our robot drawing system, we have a number of options. Advertising could show the system usage, in which a user can easily scan in plans for easy input, and let the robots run autonomously with little setup. At the same time, we could show usage of multiple robots, and provide information on the simple system scalability. The consumer product would most likely use a mobile app for providing input and status updates. For the commercial sector, the focus would be on showing drawing multiple types of lines for our various usages, such as airports, road lines, or sports fields. Finally, we believe that providing a development kit would help improve exposure of our system, allowing classrooms to leverage this robot in educational settings. 

\subsection{Technical Issues}
Building consumer-ready robot agents would require a few major technical changes as well.The largest change would be collision detection - a complex and robust methodology would be needed to ensure safety and consistent operation. The localization system would also need improvement, likely via computer vision so the robots can run regardless of location. Finally, the ability to both specify and switch between colors when drawing would be necessary for a successful product.
