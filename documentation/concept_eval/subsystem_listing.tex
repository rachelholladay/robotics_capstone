% !TEX root = main.tex

\section{Subsystem Listing}
\label{sec:subsystem_listing}

\added[remark={NJ,V2}]{The remainder of this document is split based on the various susbystems. We briefly describe each subsystem here.}

\added[remark={NJ,V2}]{Writing Implement (\sref{sec:writing_implement}): The mechanism that uses the writing tool to deposit material onto the writing surface.} \\
\added[remark={NJ,V2}]{Locomotion (\sref{sec:locomotion}): This system involves robot motion across the writing surface.} \\
\added[remark={NJ,V2}]{Localization (\sref{sec:localization}): All pieces of the robot system involved with determining position and orientation of the robot agents within the drawing space.} \\
\added[remark={NJ,V2}]{Image Processing (\sref{sec:image_processing}): Takes a user-created image and processes into a format usable by the work scheduler (\sref{sec:planning}). }\\
\added[remark={NJ,V2}]{Work Scheduling, Distribution, and Planning (\sref{sec:planning}): Determines trajectories for individual robots to complete the drawing, based on the user's input image.} \\
\added[remark={NJ,V2}]{Communication (\sref{sec:communication}): This subsystem involves all communication between robot agents and the offboard system. }\\
\added[remark={NJ,V2}]{User Interface (\sref{sec:user_interface}): A unified system for user input and interaction into the system.} \\
\added[remark={NJ,V2}]{Power System (\sref{sec:power_system}): Supplies power to all necessary parts of the robot system.}
