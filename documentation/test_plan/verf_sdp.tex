% !TEX root = system_validation.tex

\subsection{Work Scheduling, Distribution and Planning}
\label{sec:verification_sdp}

\subsubsection{Performance Test: Executable Plans}
\label{test:sdp_pt_executable}
\textbf{Test Question:} How consistent is the planner at generating executable plans, ie those that avoid collision and stay within bounds? \\
\textbf{Operational Procedure:} Given a set of example drawing inputs, run each input and check the plan for potential robot-robot collisions and out of bounds driving. \\
\textbf{Metric:} Ratio of number of unacceptable plans, those that would involve collision or driving out of bounds, over the total number of plans. \\
\textbf{Acceptance Criteria:} Almost all, 99\% of plans would not involve collision or out-of-bounds if executed. \\
\textbf{Requirement(s) Verified:} \frref{fr:in_bounds}, \nfrref{nfr:safe}

\subsubsection{Performance Test: Execution Distribution}
\label{test:sdp_pt_execu}
\textbf{Test Question:} How efficiently is execution time, i.e. the total time robots spend moving, distributed?\\
\textbf{Operational Procedure:} Given a set of example drawing inputs, run each input and record the total time each robot spends moving. \\
\textbf{Metric:} We define execution efficiency as $\frac{\min(execution(R_{0}), execution(R_{1}))}{\max(execution(R_{0}), execution(R_{1}))}$ where execution($R_{0}$) refers to the execution time of robot 0 and execution($R_{0}$) refers to the execution time of robot 1\\
\textbf{Acceptance Criteria:} Execution efficiency of 0.75.\\
\textbf{Requirement(s) Verified:} \nfrref{nfr:efficiency}, \nfrref{nfr:coordination}

\subsubsection{Performance Test: Drawing Distribution}
\label{test:sdp_pt_draw}
\textbf{Test Question:} How efficiently is drawing time, i.e. the total time robots spend drawing, distributed? \\
\textbf{Operational Procedure:}  Given a set of example drawing inputs, run each input and record the total time each robot spends drawing. \\
\textbf{Metric:} We define drawing efficiency as $\frac{\min(draw(R_{0}), draw(R_{1}))}{\max(draw(R_{0}), draw(R_{1}))}$ where draw($R_{0}$) refers to the drawing time of robot 0 and draw($R_{0}$) refers to the drawing time of robot 1\\
\textbf{Acceptance Criteria:} Drawing efficiency of 0.75.\\
\textbf{Requirement(s) Verified:} \nfrref{nfr:efficiency}, \nfrref{nfr:coordination}

\subsubsection{Performance Test: Speedup}
\label{test:sdp_pt_speedup}
\textbf{Test Question:} What speedup is achieved by using two robots instead of one?\\
\textbf{Operational Procedure:} Given a set of example drawing inputs, run each input first with one robot and then with two. Time the execution time of each variant.\\
\textbf{Metric:} The comparison of duration, i.e. $\frac{execution time with 2 robots}{execution time with 1 robot}$. \\
\textbf{Acceptance Criteria:} According to our requirements we expect a speedup of 2x. \\
\textbf{Requirement(s) Verified:} \nfrref{nfr:efficiency}

\subsubsection{Functional Test: Collision Free}
\label{test:sdp_ft_collision}
\textbf{Test Question:} Does the planner and executor generate collision free plans?\\
\textbf{Operational Procedure:} Given a set of example drawing inputs, run each input and check for any robot-robot collisions during execution. \\
\textbf{Metric:} Boolean across each plans on whether a collision occurred.\\
\textbf{Acceptance Criteria:} We only accept if collisions were avoided on 95\% of our test cases. \\
\textbf{Requirement(s) Verified:} \nfrref{nfr:safe}

\subsubsection{Functional Test: Autonomy}
\label{test:sdp_ft_autonomy}
\textbf{Test Question:} Does the system require no user input beyond adding the image to be drawn (except for error handling)? \\
\textbf{Operational Procedure:}  After having input a plan, press "Run" on the system and observe if the system requires user input to finish the drawing. \\
\textbf{Metric:} Boolean on whether user input was required, excluding input relating to errors.\\
\textbf{Acceptance Criteria:} Accept only if no input was required.\\
\textbf{Requirement(s) Verified:} \frref{fr:autonomous}


\subsubsection{Failure Mode: Fail to Plan}
\label{sec:sdp_fm_planning}
\textbf{Description:} This failure mode occurs when the offboard system is unable to generate a valid plan for the robot agents. This means the main controller is unable to command the robots to successfully complete the input drawing.\\
\textbf{Cause:} Failure to create a valid plan could arise from an out of bounds drawing. Other reasons include the image processing result being incorrect, which forces the work planner to incorrectly assign and generate plans.\\
\textbf{Effect:} The system is unable to complete an invalid drawing, and cannot begin autonomous operation.
\textbf{Criticality:} This is a system-critical failure due to the fact that the system cannot recreate the drawing if it cannot generate a robot motion plan to do so. \\
\textbf{Safety Hazards:} There are no safety hazards associated with this failure mode, given that it is entirely software-based. \\