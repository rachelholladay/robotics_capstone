% !TEX root = main.tex

\section{Use Cases}
\label{sec:use_cases}

\subsection{Reload writing implement}
Summary: The robot's main consumable, the writing implement, must be reloaded or replaced when empty or when
a different implement (with different color or stroke) is desired.
Actors:  Writing tool, human reloader
Precondition: no writing tool in the system, or one that is no longer desired
Postcondition: system has desired writing tool
Alternative: - improper loading results in robot alerting user.
             - other robot agents know broken robot cannot draw and replan accordingly
Description: When the robot is working on a large or intricate drawing that requires a large amount
of consumable writing material, the robot may not be able to carry enough material to complete its
allocation of the drawing. Additionally, some drawings may involve a variety of colors, strokes, or other
properties that necessitate the use of different writing materials. Thus, the robot must be able to
replace its writing implement with human assistance. The robot must be able to recognize reloading
failures and alert the human operator of their occurences. If the failure is not corrected, The robot must
communicate its inability to perform to partner robots so they can replan writing paths accordingly.

\subsection{Process input image}
Summary: The robot must take in a human-produced image and interpret where markings are desiredi.
Actors: image, human
Precondition:  no existing image being drawn
Postcondition: image processed and ready for work distribution between agents
Alternative: report error on image processing failure.
Description: The robot's task will be inputed using a human-produced image following specific guidelines.
From the image, the robot can determine where to place markings in the real world, and how the work should
be distributed amongst the workers.

\subsection{Localization}
Summary: The robot must be able to determine where it is in the world.
Actors: environment
Precondition: large amount of uncertainty about current location of the robot
Postcondition: little uncertainty about the current location of the robot
Alternative: report to human operator, alert other robots of its inability to perform
Description: In order to create an accurate reproduction of the input image, the robot must be able to know
how its location maps to a location on the input. If it is unable to do so, it cannot continue drawing
and must alert other robots to the fact so that they can replan.

\subsection{Scheduling and Robot Planning/Coordination}
Summary: The robot workers must determine an efficient allocation of the work.
Actors: robot workers, input image
Precondition: no plan or schedule exists
Postcondition: each robot has an allocation of work and a planned path
Alternative: A human user is alerted and the operation is aborted
Description: The work required must be determined from the input image, and analysed to determine an
efficient allocation of work, as well as a path schedule for each of the robots. Additionally,
robots must communicate work completed and failure states to each other in case replanning is required.

\subsection{Move robot}
Summary: The robot moves across flat terrain.
Actors: ground
Precondition: robot is stationary
Postcondition: robot is not stationary
Alternative: A human user is alerted and work is redistributed between remaining robots.
Summary: The robot moves across the writing surface, using its writing implement when required.

\subsection{Use writing implement}
Summary: The robot creates a mark on the writing surface
Actors: writing surface, writing implement
Precondition: no mark exists on writing surface
Postcondition: mark exists on writing surface
Alternative: Robot alerts user that the writing implement must be replaced
Summary: The robot creates markings on the surface with a variety of marking implements, and ensures
that its movements do not disturb the markings

Referring to the image below, \figref{fig:example_figure}.


\begin{figure}
 \centering
 \includegraphics[width=0.5\columnwidth]{figs/example_picture.jpg}
 \caption{This is an example caption}
 \label{fig:example_figure}
\end{figure}


