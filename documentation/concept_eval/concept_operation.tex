% !TEX root = main.tex

\pagebreak
\section{Concept Operation}
\label{sec:concept_operation}
%Rubric: "Concept of operation clearly and succinctly specifies how the exhibit will work.". Describe the system from the viewpoint of someone using it.

\begin{figure}[h!]
 \centering
  \includegraphics[width=0.90\columnwidth]{diagrams/concept_operation_user.jpg}
	\caption{State Machine of Concept of Operations}
 \label{fig:concept_operation}
\end{figure}

Above, in \figref{fig:concept_operation} we outline the user experience with the system. This is detailed below and our user interface is fully explained in \sref{sec:user_interface}.

The initial setup contains two operations that can be performed simultaneously: adding the image to be drawn to the system, and setup of the drawing surface. Adding the input image involves generating the image, and then scanning it into the system. This satisfies requirements for input the drawing plan (Requirements Specification 5.1, FR9), and a user interface (Requirements Specification 5.1, FR12). Setup of the system involves placing the drawing surface, placing and calibrating vision markers, and finally placing the robot agents within the bounds of the drawing surface. Once both steps are done, the user can enter any required settings for their use, and begin operation. Processing of the input image is done automatically by the system and is invisible to the user (\sref{sec:image_processing}).

Once the autonomous drawing process begins, the user simply observes the robots \added[remark={RH V2}]{until they} complete the task. Errors will be reported to the user, who then has the option of fixing the issue to continue operation, or terminating the drawing process. The autonomous process satisfies functional requirement FR2.
