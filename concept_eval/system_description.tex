% !TEX root = main.tex

\section{System Description}
\label{sec:system_description}

The goal of our project is to develop a multi-agent robotic painting system. \replaced[remark={RH, V2}]{Using}{Use} a team of homogenous robots, we can efficiently parallelize the dull and often dirty task of painting. Our system is motivated by the miles of paint \replaced[remark={RH, V2}]{needed}{needs} to maintain runways and sport areas as well as the widespread use of sidewalk chalk. Below, in \figref{fig:system_description}, we model our overall system \replaced[remark={RH V2}]{graphically}{graphical}, with a more detailed description following. We then enumerate each subsystem in \sref{sec:subsystem_descriptions}.

\begin{figure}[h!]
 \centering
  \includegraphics[width=0.90\columnwidth]{diagrams/systems_diagram.jpg}
	\caption{High Level Architecture Diagram}
 \label{fig:system_description}
\end{figure}

Our system encapsulates three components, shown in \figref{fig:system_description} through dotted lines. First, there is a single offboard unit that \replaced[remark={RH, V2}]{hands off most}{hands most} of the coordination, planning and user interface. The initial drawing process begins in the offboard unit, where users will generate an input image 
\deleted[remark={RH, V2}]{and process it through a scanner}. Image processing will then occur, where the input is processed into a format compatible by the planner and work distribution subsystem \sref{sec:subsystem_planner}. Moving forward, the planner module will take in updates to current robot progress and use it to update commands it sends to robot agents.

Our second component is the robotic agents, marked in \figref{fig:system_description} as "onboard". While the diagram shows one onboard agent, this schema is the same for each agent. Commands are \replaced[remark={RH, V2}]{sent}{sednt} to these agents.  Communication protocols are able to send information both to and from the work planner to robot agents (\sref{sec:subsystem_communication}). From here, onboard systems are separated into three subsystems: Localization (\sref{sec:subsystem_localization}), Locomotion (\sref{sec:subsystem_locomotion}), and Paint Distribution (\sref{sec:subsystem_writing_implement_actuator}). The paint distributor is involved with all commands regarding engaging and disengaging the writing implement to output paint onto the writing surface. Locomotion is the drive system, which engages wheels and motors used for movement. This subsystem also includes any motion-related safety mechanisms, such as quickly stopping. The Localization subsystem works to determine the position and orientation of an individual robot agent. This subsystem takes input from the sensor module (\sref{sec:subsystem_sensors}), which provides data about the surroundings for the localization system to process.

The sensor module gets data from external vision markers (\sref{sec:subsystem_markers}) for processing by localization. It is important to note that the sensor module, along with the locomotion and paint distribution subsystems are all powered by an individual subsystem \sref{sec:subsystem_power}.

Finally our third component comes from environmental setup, which is mainly the situational landmark tools need to localize the robot.  