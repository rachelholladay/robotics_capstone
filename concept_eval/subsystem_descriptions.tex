% !TEX root = main.tex

\section{Subsystem Descriptions}
\label{sec:subsystem_descriptions}

\rhnote{Rubric: All subsystem descriptions provide a clear idea of each subsystem and any critical component is identified.}

\rhnote{Some subsystems might need their own diagram}

\rhnote{Here are possible subsystems}

\subsection{Paint Distributor}
\subsection{Locomotion} 
drives individual robot. want to go sideways. have enough clearence for writing tool. based on trade studies need wheels. 

\subsection{Localization}
take in input from sensory module and determine accurate position and orientation of individual robot. communicates through communicator to scheduler. 

\subsection{Input Image Scanner}
put in input image. probably take or generate picture then handoff to image processor. critical component: high enough resolution sensor to capture image. has user interaction. 

\subsection{Image Processor}
needs to process image. take image scanned by user and parse it into input regonizable by work scheduling, distribution and planning

\subsection{Work Scheduling, Distribution and Planning}
see software architecture. couple small senteces. 

\subsection{Communication}
antenna fit within requirements. have certain range. allow multi agent communication. want to be fairly speedy and realiable. possible communication protocols: bluetooth, wifi, serial communication (like mav link),  

\subsection{User Interface}
Insert flow chart for box "specify drawing settings" from concept operation

\subsection{Sensor Module}
interface to vision markers and localization. Allows for localization. Also needs to be small enough to fit in footprint. 

\subsection{Power System}
Needs to onboard power. probably battery. Battery needs to small and light enough to satisfy the size and weight requirement. 

\subsection{Vision Markers}
Fixed. Put markers on all 4 sides. Calibrated in initialization. Reference trade study