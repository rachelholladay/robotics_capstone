% !TEX root = main.tex
\section{System Overview}
\label{sec:system_overview}

This section details individual subsystems and how they function together to form the complete robot system. It is split into descriptions of the mechanical design, electrical components, and software subsystem architecture. System design is described for a single robot, with some software subsystems utilizing information for both robots at once.

\subsection{Mechanical System}
\label{sec:mechanical}
As shown below in \figref{fig:so1}, our system contains two identical robots. Each robot is 23 cm in diameter, 19 cm tall, and weights 1.5lb with all electronics installed. Details for mechanical components are explained below. Besides the two robots, mechanical system also contains a camera jig, constructed using 8020 aluminum extrusions. 

\begin{figure}[h!]
\centering
\includegraphics[width=0.98\columnwidth]{figs/full_system.png}
\caption{Robots Overview}
\label{fig:so1}
\end{figure}

\subsubsection{Locomotion}
\label{sec:me_locomotion}
For locomotion system, each robot drives with four Mecanum wheels. Due to the nature of drawing, we expect our robots to turn in sharp corners quite often. Therefore, Mecanum wheel become the best locomotion mechanism, for its high mobility and high controllability. As shown in the photo of wheel assemble (Figure… left), each Mecanum wheel is powered by a metal micro gearmotor which is mounted to the chassis via an off-the-shelf motor bracket. Mecanum wheel is connected to the motor through a customized wheel adaptor, as shown in Figure… right. These wheel adaptors were fabricated using 3D printer so that we could optimize through many design iterations in a short period of time.


\subsubsection{Writing Tool}
\label{sec:me_writing}
The robot writes using a lever-like writing mechanism. The chalk marker is slipped into a customized marker holder (Figure.. left) which has a tight D-shaft cutout for connecting the micro gearmotor. As the motor rotates, the marker holder would lift or drop the marker which creates or discontinues drawing lines. The assembled mechanism in shown below in Figure .. middle. Again, the customized marker holder is fabricated using 3D printer for rapid prototyping purpose. 

Compared to other drawing utilities, like spray paint, wet paint, or regular markers, we used chalk based drawing tool because, in this way, we can easily reuse drawing surface and its finish mark has unique look and texture. We choose liquid chalk marker, instead of regular chalk, because we do not need to account for the issue brought by drawing utility shortens as being consumed. 

\subsubsection{Chassis}
\label{sec:me_chassis}
Since we are mounting a lot of electronics and mechanical components onto the chassis and the whole robot is rather lightweight, we choose laser cutting as the fabrication technique for producing the chassis (Figure.. left). Chassis is laser cut using acrylic because it has a good combination of durability and machinability. 

Supporting structures for Raspberry Pi and April Tag are also assembled using laser cut acrylic pieces. They are attached to the chassis through slot joints, as shown below in Figure … right. Batteries for Raspberry Pi and motors are attached to the chassis using Velcro tapes so that we can easily switch them. 

\subsubsection{Infrastructure}
\label{sec:me_infrastructure}
As shown in Figure…, the camera jig is designed and machined using 8020 aluminum extrusion which are chosen due to the frames’ ease-of-use and durability. The jig has a footprint of 6ft by 6ft and a height of 5ft.  5ft is determined to the optimal height for capturing all corners of the 6ft by 6ft work space. The camera will be mounted to the jig using Velcro tape. Due to 80/20 structure’s easy construction, we can simply disassemble the jig into two pieces for transportation. Therefore, despite its large size, the camera jig is not restricted to any specific space. 

The work space is constructed by laying chalk board paper on sheets of plywood board. Using plywood ensures that the robots always operate on flat surface. Due to the size constrain of available plywood sheets, the work space is assembled using two 3ft by 6ft sheets. These two sheets can then be folded for ease of transportation. 

\subsection{Electrical System}
\label{sec:electrical}
The electrical systems for both robots are identical. A Raspberry Pi is used as the main controller, with individual motor controllers for each of the four wheels, and the writing implement motor. Power is provided to the Raspberry Pi via power bank, and the motors have separate power provided via a series of batteries. Use of battery power for the Raspberry Pi and motors allows for each robot to be completely autonomous, without the need for external power during operation.

\subsection{Software System}
\label{sec:software}
The software system is mainly run offboard, where most system processing occurs. The onboard software systems process directional commands to direct the motors for movement and controlling the writing implement.

\subsubsection{Controller}
\label{sec:sw_controller}
The controller takes information from all software subsystems, both onboard and offboard, and combines it to command the system as a whole. The offboard controller takes waypoint input from the planner, as described in \sref{sec:sw_sdp}, and combines it with localization data to determine each robot's next action. This includes collision response, and standard locomotion targets. This data is processed and combined into a message sent via the communication subsystem to the onboard system. The onboard system is significantly simpler: it waits for new messages from the offboard system, and upon receiving, processes the target into motor commands which are then directly sent to each motor. Motor commands include wheel locomotion and movement of the writing implement.

At a fixed interval of 20Hz, the offboard controller runs an update loop. This update loop contains all processing, which pulls data from the various subsystems as described above. All blocking subsystems and functions are processed in separate threads, to minimize latency between messages sent to the robots, and to maintain a constant update loop speed.

\subsubsection{Communication}
\label{sec:sw_comm}
The communication subsystem has been static in design throughout the project. As input, it receives data consisting of locomotion and onboard processes, and sends them over TCP wireless connection. Locomotion data is simplified for each robot. It contains the robot's current transform (2D position and orientation), and target waypoint transform. It also includes a stop parameter, and the expected status of the writing implement. These parameters are packaged into the Protobuf message format, which is serialized and passed wirelessly. The onboard system takes advantage of the communication subsystem by polling for new messages, which provide updates to the robot's mechanical status. Processing of these messages are done by the onboard controller, and the locomotion subsystem (\sref{sec:sw_locomotion}).

Messages are processed and sent asynchronously from the offboard controller. The subsystem holds partially completed messages as they are constructed in a single update loop, and then a message thread is activated upon receiving a message-ready status. Each robot has its own message thread, which will send the message while the controller continues on to its next update loop.

\subsubsection{Locomotion}
\label{sec:sw_locomotion}
Locomotion subsystem processing is mainly done onboard each individual robot. At each onboard update loop, a message with the robot's current transform and target transform is computed. The vector between them is used to compute mecanum wheel motor powers. Translational motion uses the planar 2-dimensional vector, using standard mecanum wheel equations with a calibrated speed. Rotational motion is used as a correction factor, to help the robots maintain a forward heading throughout operation. When the robot's current orientation exceeds a threshold over the desired forward heading, a slow rotational factor is accumulated to the mecanum wheel powers to correct heading. This correction is purposefully slow, to prevent it from overshadowing translational motion, which could result in curved motions when straight lines are preferred.

\subsubsection{Localization}
\label{sec:sw_localization}
Robot localization is done via the offboard system, which parses data from the overhead webcam to track Apriltags. Apriltag detection is done for the four corners of the drawing space, and for the robots, via a C++ Apriltag library. Data is passed up to the main Python control layer using Boost Python, and returned to the main controller. Apriltag data is received and oriented back from the webcam's pixel coordinate frame to a coordinate space relative to the corner Apriltags. This coordinate space transformation is necessary to ensure the robots can continually orient themselves with a forward heading.

Similar to other subsystems, localization is run in a separate thread from the main controller. When the controller calls for localization data, the most recently processed robot transform data is returned. Localization runs at a consistent 30Hz.

\subsubsection{Writing Tool}
\label{sec:sw_writing}
Writing tool control is done via the onboard systems. The offboard controller passes an enable or disable status to the onboard system, which then controls the tool. To ensure the tool maintains a consistent force on the ground, it will send a pulsing signal to the motor when enabled. This ensures that the tool is always on the ground, even when small bumps or uneven terrain pushes the tool upwards during operation.

\subsubsection{Scheduling, Distribution, and Planning (SDP)}
\label{sec:sw_sdp}
\rhnote{TODO}


