% !TEX root = main.tex

\subsection{Power System}
\label{sec:hardware_power}

\subsubsection{Fabrication Procedure}
\label{sec:power_fab}
Power system contains a Raspberry Pi and two power supplies. All of them will be purchased online.

\subsubsection{Assembly Instruction}
\label{sec:power_assemb}
1. Tape Velcro strips on the chassis, the Raspberry Pi, and the batteries.
2. Attach the Rasberry Pi and the batteris to the chassis via Velcro. 

\subsubsection{Electronics}
\label{sec:power_electronics}
The Raspberry Pi will draw its power from a 5V USB power bank. The battery pack can hold 3400 mAh of charge, which should allow more than 2 hours of operation. The battery packs are easily swappable and cheap, meaning that our budget allows for multiple packs. Additionally, battery packs can charge each other, making it uncessary to power down the Raspberry Pi when additional charge is needed.

A separate locomotive battery pack will provide power to the motors. This battery pack contains 8 rechargable AA batteries, giving 12V and containing 2000 mAh of charge. This allows around 30 minutes of continuous operation for the motors, requiring only a quick battery swap when charge is depleted.