% !TEX root = main.tex

\section {Executive Summary}
\label{sec:executive_summary}
This report provides a requirements specification for our robotics capstone project. Through this document we aim to sufficiently and suciently outline our project. 

\subsection{Project Overview}
The goal of this project is to build a multi-agent system that collaboratively and efficiently recreates inputted images at variable scale. 
This system can be used for either reproducing works of art on a larger scale for aesthetic purposes or for marking elements of infrastructure. 
By using a team of robots that work together, as opposed to a single robotic system, we hope to gain greater efficiency as well as explore various coordination schemes. 


\subsection{Document Outline}
Following this executive summary, we begin by outline the purpose of our project, including its goals (\sref{sec:project_goal}) and motivation (\sref{sec:motivation}). 
We detail both our intended scope in this project (\sref{sec:project_scope}) and assumptions we make about our environment and operation (\sref{sec:assumptions}). 

Next we review the requirements of our system (\sref{sec:requirements}). 
We begin by identifying our functional requirements (\sref{sec:functional_requirements}), which define the functionality of our system. 
Next we detail our non-functional requirements (\sref{sec:nonfunctional_requirements}), which provide us with testable performance benchmarks.

We then identify scenarios where our system could be deployed (\sref{sec:scenarios}). This various scenarios informed our requirements. \rhnote{Change order to put requirements last?} Finally, we describe the various use cases of our system (\sref{sec:use_cases}). 
