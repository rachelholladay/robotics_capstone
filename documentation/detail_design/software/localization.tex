% !TEX root = main.tex

\subsection{Localization}
\label{sec:software_localization}
Localization uses data gathered from real-time overhead-mounted camera data to localize the drawing area bounds, and robot positions and orientations. Important locations are marked by AprilTags. AprilTags are the equivalent of 2D barcodes, which can be calibrated and detected at range to determine their orientation and position \cite{apriltags}. Localization software will use the AprilTags C++ library. In order to provide compatibility with the Python scripts used for locomotion and control \sref{sec:software_locomotion}, Boost Python will be used to wrap C++ functions into the Python layer \cite{python_boost}.

The drawing boundary, which is static throughout the course of a single drawing, will be represented by four AprilTags placed at the four corners. During the localization loop, these will be found and will remain the same at every iteration. By detecting all four corners, the maximum and minimum coordinates of tags in camera image space can be found. This will then define the bounds for computing locomotion commands. For the bounds, the orientation of the tags make no difference.

In contrast, tracking the robot agents requires both position and orientation tracking. Positions will be reported relative to the bounds, with the bottom left corner bounds marker being designated as (0,0). Similarly, orientations will be reported as an angle relative to the base of the detected bounds.
