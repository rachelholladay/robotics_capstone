% !TEX root = main.tex

\section{Trade Studies}
\label{sec:trade_studies}

In creating our multi-agent drawing robot, we face design trade-offs and decisions for many of our major subsystems detailed in \sref{sec:subsystem_descriptions}. We investigate three major design decisions in the following brief trade studies. Our discussions lead to a few technical solutions, when we feel we have enough data and strong reasoning to back them up. However, for some of our discussion points, further prototyping and testing are needed to fully evaluate all options. For these points we lay out brief prototyping plans below. 

\subsection{Localization Method}
\label{sec:trade_localization}
In multi-agent planning, it is important to accurately localize robots' positions and orientations. Keeping in mind limitations in price and ease of use, we come up with two major methods for localization: vision-based and marker based. They are described below.

Vision-based localization involves using cameras or other visual sensors to directly obtain information of the environment and localize the robots in the system based on found landmarks in that environment. One example of this is SLAM (Simultaneous Localization and Mapping), often used by autonomous vehicles to simultaneously build maps and localize ~\cite{dissanayake2001solution}. With this approach, robots could build small maps of their surroundings and match their locations to features they find in the environment. Benefits of this method include being location agnostic and requiring no additional parts or external setup. However, pure vision systems are difficult to calibrate and localization accuracy can depend heavily on static surroundings, which is not something this system can guarantee.

The other choice of methodology is marker-based. Using markers placed around the drawing surface, robot agents can quickly locate these markers and their positions relative to each marker, and consequently triangulate their positions and orientations. While requiring additional setup and more parts to calibrate than vision-based localization, existing technology makes it convenient and cheap to get marker-based localization working quickly. One example of a marker-based localization system is AprilTags~\cite{olson2011apriltag}, which can be described as 2D barcodes placed in a scene. Marker-based localization can be further classified into two subcategories: passive and active. Passive markers do not output any information and exist for the robot agent to observe and triangulate accordingly. AprilTags is an example of the passive marker system. On the other hand, active markers will \"look at\" robot agents to determine where the agents are, rather than the robots searching for markers. While less common, active systems behave well in conditions when the markers may not always be easily visible to robot agents~\cite{cassinis2005active}.

Given the convenience and ease of use of marker-based localization, it is clearly the better choice. However, it is more difficult to determine whether using active or passive systems will be more effective. A prototype for each system should be made for further evaluation.

\subsection{Locomotion Method}
\label{sec:trade_locomotion}
One of our functional requirements with the highest priority (FR1) is to be able to move in specified directions with a high degree of accuracy. Additionally, a medium high priority functional requirement (FR 9) is to have a drive control system that enables this movement. Therefore, it is clear from our requirements that locomotion is critical to our robot's performance. When considering locomotion options there are three large categories: aerial, legged and wheeled. 

Our drawing occurs on the two dimensional plane. Therefore, any flight based system would have to constrain one translational degree of freedom and two rotational degrees of freedom. Doing so would require precise control algorithm and large power input. Practically speaking, this seems to over-complicate the problem with little additional gain. 

Therefore, we are narrowed down to a legged or wheeled system. While a legged system would have the ability to traverse uneven terrain, such capability is not necessary for our project due to our assumption (A3) that the drawing surface is flat and homogenous. Furthermore, compared to wheeled systems, legged systems usually involve more complicated control algorithms and more complex electrical and mechanical components. We are inspired by the wide spread success of wheeled robotic systems and especially appreciate their stability, a critical concern when carefully drawing images.

Therefore, we have concluded that our agent's drive system should be wheeled. In our next analysis, we will further investigate different types of drive train. 

\subsection{Drive System}
\label{sec:trade_drive}

\yjnote{I think we could include more details and compare more options, like elaborating tracks and skid drive in the four wheel section. I can work on this tomorrow.}

In continuing to prioritize our functional requirement (FR1) of being able to accurately move in specified directions, we further investigate possible drive systems. 

Three similar drive systems first come to mind: differential drive, ackerman steering and four-wheel steering. Differential drive on a robot typically consists of two independently driven wheels and a non-driven wheel. Moving up in complexity and taking after some cars is ackerman steering, where the back pair of wheels is fixed in orientation but the front pair can pivot. In more complex is four-wheel steering, which is similar to ackerman steering but allows the back wheels to also pivot. However, these systems suffer from the fundamental issue of being nonholonomic. Therefore, while they could achieve full mobility, this might complicate drawing and path planning. 

Therefore instead we want an omnidirectional base that is holonomic. This can be achieved with omniwheels, perhaps more commonly, with mecanum wheels. 


%Rubric: Conduct trade studies when there are significant alternatives. Describe trade studies (use comparative analysis techniques) and results. Trade studies present for all major design decisions and clearly identify reasons for choices. If no choice is made, a prototyping plan is identified.
%According to google: "A trade study or trade-off study is the activity of a multidisciplinary team to identify the most balanced technical solutions among a set of proposed viable solutions (FAA 2006). These viable solutions are judged by their satisfaction of a series of measures or cost functions."
