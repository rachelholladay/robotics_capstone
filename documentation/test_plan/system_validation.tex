% !TEX root = main.tex

\section{Full System Validation}
\label{sec:system_validation}

\subsection{Performance Test: Painting Accuracy}
\label{test:sys_pt_accuracy}
\textbf{Test Question:} How closely does the drawn image resemble the original image?\\
\textbf{Operational Procedure:} \replaced[remark={RH, V2}]{Using the example input set (\appref{app:planner_inputs}), input each }{Input a sample image} for the system to complete. After completion, overlap the original image with the image of final drawing captured from overhead camera. Rescale the two images so that they are in the same size. Evaluate the coherence of the two images.\\
\textbf{Metric:} The percentage of drawn lines that were within 3 pixels of difference compared to those of the original image.\\
\textbf{Acceptance Criteria:} The system must successfully and accurately draw 95\% of the lines in the original image.\\
\textbf{Requirement(s) Verified:} \nfrref{nfr:quality}

\subsection{Performance Test: Reliability}
\label{test:sys_pt_reliability}
\textbf{Test Question:} How reliable is the system in terms of successfully complete a series of drawing tasks?\\
\textbf{Operational Procedure:} Command the system to finish \replaced[remark={RH, V2}]{the example input set (\appref{app:planner_inputs}).}{a series of drawing tasks.} Measure the number of consecutive successful completion. Successful completion is defined as the system autonomously finishes painting and the painting process is free of errors including but not limited to localization breakdown, motor breakdown, or painting mechanism breakdown. Calling human interference with switching battery and drawing utility does not count as unsuccessful run.\\
\textbf{Metric:} Number of consecutive painting completion. \\
\textbf{Acceptance Criteria:} The minimum acceptable number of consecutive completion is 5.\\
\textbf{Requirement(s) Verified:} \nfrref{nfr:reliability}

\subsection{Functional Test: Size}
\label{test:sys_ft_size}
\textbf{Test Question:} Is the robot agent too big to be portable, i.e. carry the robot through a standard door?\\
\textbf{Operational Procedure:} Measure the physical dimensions of the robot in terms of width, length, and height or in terms of diameter and height. \\
\textbf{Metric:} Numeric value of each length measurement; robot footprint; robot volume.\\
\textbf{Acceptance Criteria:} Must be less than 80 in. x 36 in. x 36 in. \\
\textbf{Requirement(s) Verified:} \nfrref{nfr:size_limit}

\subsection{Functional Test: Weight}
\label{test:sys_ft_weight}
\textbf{Test Question:} Is the robot agent too heavy to be portable, i.e. able to be lifted by a normal person?\\
\textbf{Operational Procedure:} Measure the mass of the robot. \\
\textbf{Metric:} Numeric value of robot mass.\\
\textbf{Acceptance Criteria:} Must be less than 50 pounds. \\
\textbf{Requirement(s) Verified:} \nfrref{nfr:weight_limit}

\subsection{Functional Test: Budget}
\label{test:sys_ft_budget}
\textbf{Test Question:} Does the cost for developing this robotic system exceed our budget?\\
\textbf{Operational Procedure:} Document total amount of money spent for designing and constructing this robot system. This includes machining expense, part cost, and etc. \\
\textbf{Metric:} Total amount of money spent.\\
\textbf{Acceptance Criteria:} Total developing expense has to be less than \$2500. \\
\textbf{Requirement(s) Verified:} \nfrref{nfr:budget}

\subsection{Functional Test: Safety}
\label{test:sys_ft_safety}
\textbf{Test Question:} Is the robot safe during operation? Specifically, when collision happens, will the robot harm other robots, external environment, or human?\\
\textbf{Operational Procedure:} Count the number of sharp edges on the exterior of the robot. Also, measure the time it takes from the overhead camera detects collision to robot agent stops moving motors. Intermediate steps involved are: camera sends collision signal to system controller and system controller commands involved robot agent to stop its current action. \\
\textbf{Metric:} Number of sharp edges; amount of time takes from detection to action.\\
\textbf{Acceptance Criteria:} Values for these two metrics need to be as small as possible. The maximum number of sharp edges on the exterior is 4, assuming a rectangular chassis design. The maximum amount of time is 1.5 seconds. \\
\textbf{Requirement(s) Verified:} \nfrref{nfr:safe}

\subsection{Functional Test: Documentation}
\label{test:sys_ft_Documentation}
\textbf{Test Question:} Is the documentation for the developing process comprehensive and replicable?\\
\textbf{Operational Procedure:} Give the full documentation to another design group or stakeholder and inquiry if they can duplicate the project with those documents.\\
\textbf{Metric:} Boolean on whether or not reviewers can replicate the system development.\\
\textbf{Acceptance Criteria:} We must succeed on the account.\\
\textbf{Requirement(s) Verified:} \nfrref{nfr:documentation}

\deleted[remark={RH, V2}]{\textbf{Failure Mode: Out of Bounds}}
\deleted[remark={RH, V2}]{\textbf{Description:} This failure describes the situation when any robot agents move beyond the bounds of the drawing surface, as described by the boundary vision markers.}
\deleted[remark={RH, V2}]{\textbf{Cause:} This error can be caused by either poor localization, or poor locomotion. If localization software believes the robot to be somewhere it is not, it may command the robot out of bounds. If the robot motors or wheels are not working properly, it may move out of bounds, despite being given correct motion commands.}
\deleted[remark={RH, V2}]{\textbf{Effect:} The robot moving out of bounds introduces undefined behavior that could result in collision, incorrect drawings, or making marks with the writing tool that are not on the appropriate writing surface.}
\deleted[remark={RH, V2}]{\textbf{Criticality:} This is a critical error, as it results in incorrect localization, drawing marks, and system operation. Entering this failure mode results in system operation ending.}
\deleted[remark={RH, V2}]{\textbf{Safety Hazards:} This failure poses a safety hazard of collision, as robots that exist the bounds may collide with objects or people that it is not expecting.}
\deleted[remark={RH, V2}]{\textbf{Failure Tree:} See \figref{fig:out_of_bounds_failure}}

\deleted[remark={RH, V2}]{\textbf{Failure Mode: Incorrect Markings}}
\deleted[remark={RH, V2}]{\textbf{Description:} Incorrect markings are made when a robot agent lowers the writing implement, and makes a mark in a location that does not match with the input drawing. }
\deleted[remark={RH, V2}]{\textbf{Cause:} The cause of incorrect markings could be a result of the writing implement, locomotion, or localization subsystems. The writing implement subsystem may malfunction and lower the tool at an incorrect time. The locomotion subsystem may move the robot incorrectly while the writing implement is lowered, making markings where they are not expected. Localization error can cause markings to be in places that the system thinks are correct, but do not line up with the input drawing.}
\deleted[remark={RH, V2}]{\textbf{Effect:} The effect of this failure is that the output drawing by the system is incorrect.}
\deleted[remark={RH, V2}]{\textbf{Criticality:} Given that the goal of this robot subsystem is to accurately recreate an input drawing, failure to do so is a critical error.}
\deleted[remark={RH, V2}]{\textbf{Safety Hazards:} There are no safety hazards associated with incorrectly marking the drawing surface.}

\deleted[remark={RH, V2}]{\textbf{Failure Mode: Robot Collision}}
\deleted[remark={RH, V2}]{\textbf{Description:} This failure exists when robot agents collide with any obstacle, including each other or external obstacles.}
\deleted[remark={RH, V2}]{\textbf{Cause:} Robot collision is a result of the locomotion, localization, or work scheduling subsystems failing. Locomotion failure could cause undefined motions by a robot agent, causing it to hit another robot, or move out of bounds (\sref{sec:sys_val_fm_bounds}) and collide with an obstacle. Localization failure could result in incorrect motion commands, resulting in collision with unintended objects. Finally, poor motion planning has potential to require the robots to move into collision with each other.}
\deleted[remark={RH, V2}]{\textbf{Effect:} Robot collision could damage the robot agents, or hurt human observers who are hit.}
\deleted[remark={RH, V2}]{\textbf{Criticality:} This failure is of medium importance. The robots are designed to be safe (\nfrref{nfr:safe}), and therefore are unlikely to hurt or be significantly damaged by a collision. }
\deleted[remark={RH, V2}]{\textbf{Safety Hazards:} There is a hazard of minor human injury, but as stated above the robots are designed to minimize human injury in the case of collision.}

