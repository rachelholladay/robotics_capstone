% !TEX root = main.tex

\subsection{Communication}
\label{sec:software_comm}

The communication software subsystem handles sending information to and from the robot agents and the offboard system. The offboard system will be tasked with mainly with sending locomotion and emergency commands to the robot agents. The robot agents will send logging information and odometry measured from motor actions back to the offboard system for processing. The offboard system can use received data to determine if a robot agent has fallen into an error state.

\subsubsection{Libraries \& Protocols}
\label{sec:software_comm_libs}
Communication will be handled via a TCP connection between a robot agent and the offboard system. Given that the robots only receive and process commands relating to their own motion, there is no need for the robots to be able to directly communicate with each other. 

Once a TCP connection is established, data will be sent using Protocol Buffers, a Google-designed standard for serializing structured data \cite{protobuf3}. For this project, Protocol Buffers language version 3 (proto3) will be used due to its improved speed and features over the previous version, proto2. 

Using proto3 will allow the offboard system to send commands in packets, called \'messages\'. This way, the system can organize locomotion and emergency commands into separate messages or sections of a message, which can then easily be parsed by the robot agents. 

\subsubsection{Message Design}
\label{sec:software_comm_msg}
Message design will be addressed first with regard to messages sent to robot agents, then messages sent from robot agents to the offboard system.

All messages sent to the robot agents will involve emergency information commands or locomotion. Emergency commands can be separated into full system stop, and system pause. System stop is the emergency kill switch, which occurs by user command or when the system encounters a fatal error. This will end onboard robot operation. System pause occurs for errors that are recoverable; an example of this would be robot collision. The user has the option to assess and correct any issues, and resume operation. Both system stop and pause will halt all locomotion commands until the robot agent is commanded otherwise.

Locomotion commands will be continually sent to robot agents. These commands will come in the form of relative motor powers to command each of the four mecanum wheels. These messages will be sent continually, to allow for stable corrections to robot motion to maintain smooth lines, which increases the overall quality of drawing.

Robot agents are responsible for sending odometry and logging information back to the offboard processor. Odometry information is sent for each of the four motors on the robot, by sending the encoder ticks since the last communication. This odometry information can be incorporated into localization. Motor encoder movement for the writing implement motor will also be sent. Logging information can be used for debugging or system analysis. An example of logging information is the robot battery level.

