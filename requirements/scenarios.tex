% !TEX root = main.tex

\section{Scenarios}
\label{sec:scenarios}
Our drawing robotic system has many use cases, three of which are detailed below. These scenarios then informed our use cases, described in \sref{sec:use_cases}. These are the scenarios that motivate us to work on this project and are the scenarios that a fully developed multi-agent autonomous robotic drawing system is expected to achieve. However, for the scope of this class, we are only focusing on the chalk drawing scenario, as detailed in \sref{sec:chalk_drawing}.
\rhnote{Need to say that only planning on doing chalk drawing one, but a fully fleshed out system would do all of this}
\rhnote{Remove some fluff from these}

\subsection{Chalk Drawing}
\label{sec:chalk_drawing}
Chalk drawings are often used around campuses and different communities for aesthetic purpose, information sharing, and events announcements, as seen in \figref{fig:chalk}. However, these drawings are often limited by size and complexity of the drawing. Therefore, we plan to design our robotic system to draw large-scale items on blacktop or asphalt surfaces with chalks.

Since chalk will be the main painting tool in this scenario and these robots may work outdoors, we need to design the system to prepare for situations like drawing on relative wet surfaces and protecting chalk from rain or excessive humidity. We also need to design the system to accommodate the fact that chalk becomes shorter as it is used. 

\begin{figure}
 \centering
 \includegraphics[width=0.24\columnwidth]{figs/chalk0.jpg}
 \includegraphics[width=0.24\columnwidth]{figs/chalk1.jpg}
 \includegraphics[width=0.24\columnwidth]{figs/chalk2.jpg}
 \includegraphics[width=0.24\columnwidth]{figs/chalk3.png}
 \caption{Various Examples of outdoor chalk drawing for pure fun or more serious announcements.}
 \label{fig:chalk}
\end{figure}

\subsection{Infrastructure Lane Drawing}
\label{sec:infrastructure_drawing}
Drawing infrastructure lanes, such as those for parking lots, highways, streets, and airports, can often be dull and expensive. For example, in \figref{fig:airport}, we can see the massive amount of routing lines that airport runway systems rely on. However, these lines are usually painted via a human operator. Therefore, we have the opportunity to improve this task through our robotic system. 

Since the drawn lanes need to be smooth and consistent, motor control becomes critical in this scenario. Completing this scenario requires working outdoors. Therefore, the robotic system needs to be more robust both in hardware and in software. The mechanical parts need to be weather resistant to a certain extent and the programming scripts need to account for unpredictable situations when working outdoor. 

\begin{figure}
 \centering
 \includegraphics[width=0.48\columnwidth]{figs/airport_layout.jpg}
 \includegraphics[width=0.48\columnwidth]{figs/airport_painter.png}
 \caption{Airports rely on gridlines (left), which are currently painted and repainted by human operators (right).}
 \label{fig:airport}
\end{figure}


\subsection{Sport Lines}
Drawing lanes for sports fields, including soccer fields, football fields, basketball court, etc, could be another scenario for this robotic system. Similar to our infrastructure example, these lines are quite repetitive and must be redrawn often to ensure a quality field. 

When painting certain sports fields, like soccer fields or football fields, these robots need to travel on uneven surfaces, such as grass. Therefore, better suspension and drive system will be designed to complete these tasks. Similar to the infrastructure lane drawing scenario (\sref{sec:infrastructure_drawing}), the robotic system needs to guarantee the quality of drawing lanes and being weather resistant. 
