% !TEX root = main.tex

\section{Requirements}
\label{sec:requirements}
We outline our systems requirements, both functional and nonfunctional. For each requirement we provide a number, for each of reference, a short description and a longer explanation. We prioritize our system requirements on a Likert scale from 1 to 7, detailed below. 

\CheckTable{1}

\subsection{Functional Requirements}
\label{sec:functional_requirements}

\begin{functional_requirement}{Move in Specified Directions}{7}
\item Robot must be able to autonomously move in a commanded direction on a flat plane. Omnidirectional movement is necessary to ensure robot agents can adequately and efficiently cover the drawing workspace.
\end{functional_requirement}

\begin{functional_requirement}{Autonomous}{6}
\item Given an input image to draw, robot agents must be able to autonomously complete the drawing. This includes the following steps: Processing the input, planning and commanding individual agents, and having robots move and draw without external input.
\end{functional_requirement}

\begin{functional_requirement}{Robots Localize Globally and Locally}{7}
\item All drawing robots can determine their location and orientation on a global and local scale. Global scale is relative to the localization markers and the specified drawing surface. Local scale is relative to other robot agents. This requirement is necessary so the robot system can coordinate planning together to avoid collisions or an inefficient spread of work.
\end{functional_requirement}

\begin{functional_requirement}{Safe}{4}
\item Robots must maintain safety with respect to each other, and the external world at all times. This requires all robot agents must avoid collisions. Robot agents must have an enforced maximum speed limit to avoid damage to themselves or human bystanders in case of collision.
\end{functional_requirement}

\begin{functional_requirement}{Within Bounds}{3}
\item While in operation, all robot agents must stay within bounds of the workspace. This is important to minimize external collisions, as well as to ensure localization maintains accuracy while drawing occurs.
\end{functional_requirement}

\begin{functional_requirement}{Change Writing Tools}{2}
\item Inserting, removing, and replacing writing implements must be convenient and fast. 
\end{functional_requirement}

\begin{functional_requirement}{Drive Control System}{5}
\item Robot agents must have a controller to ensure motions made are accurate. Accurate motions is paramount to ensuring an accurate drawing.
\end{functional_requirement}

\begin{functional_requirement}{Turn on or off writing tool}{1}
\item All robot agents need to be able to enable or disable use of the writing implement. The robot must also have the ability to move safely and accurately regardless of the state of the writing implement. Not all drawings are contiguous lines, and as such the robots must be able to disable the writing tool to move to a new drawing location.
\end{functional_requirement}

\begin{functional_requirement}{Input drawing Plan}{6}
\item The main controller system must be able to receive an input that allows it to command the robot agents to draw an appropriate image. The system must be able to parse the input into a state usable for robot planning and control in order to draw the input.
\end{functional_requirement}

\begin{functional_requirement}{Robots Know Progress}{2}
\item All robot agents are required to understand how much of the drawing each one has completed, and which sections are left to be drawn. This is necessary to ensure an equal spread of workload across all robot agents.
\end{functional_requirement}

\begin{functional_requirement}{Kill Switch}{3}
\item Human bystanders must be able to end all robot operation instantaneously with a kill switch or power button. This is necessary to ensure that the system can be shut down in case of an unsafe error or problem.
\end{functional_requirement}

\begin{functional_requirement}{User Interface to robot}{1}
\item An intuitive an useful user experience is necessary for efficient usage of the system. Having a simple to operate system also reduces the likelihood of user error during the input or operation stage. For industrial or commercial applications, accessibility becomes important as well. 
\end{functional_requirement}

\begin{functional_requirement}{Be In Budget}{7}
\item Design and implementation of this robotic system is limited by budget, which must be strictly adhered to.
\end{functional_requirement}

\begin{functional_requirement}{Documentation}{3}
\item Documentation of the design process, software, and hardware implementation is important for debugging, recreation, and general understanding of this project.
\end{functional_requirement}


\subsection{Non-Functional Requirements}
\label{sec:nonfunctional_requirements}

\begin{nonfunctional_requirement}{Portable}{1}
\item Small, can be carried, easy to move around. Weight less than 50 pounds. Size: bigger than 2 foot cube
\end{nonfunctional_requirement}

\begin{nonfunctional_requirement}{Completes Task in Timely manner}{1}
\item Details
\end{nonfunctional_requirement}

\begin{nonfunctional_requirement}{Quality}{1}
\item Matches input well. 
\end{nonfunctional_requirement}

\begin{nonfunctional_requirement}{Mobile App}{1}
\item Neil go nuts on your bullshit
\end{nonfunctional_requirement}

\begin{nonfunctional_requirement}{Reliability}{1}
\item Percent up time
\end{nonfunctional_requirement}

\begin{nonfunctional_requirement}{Battery Life}{1}
\item Needs to last
\end{nonfunctional_requirement}

\begin{nonfunctional_requirement}{Fault Tolerance}{1}
\item Needs to last
\end{nonfunctional_requirement}

\begin{nonfunctional_requirement}{Coordination}{1}
\item dont duplicate work or overlap
\end{nonfunctional_requirement}

\begin{nonfunctional_requirement}{Efficiency}{1}
\item split up work evenly
\end{nonfunctional_requirement}
