% !TEX root = main.tex

\section{Full System Validation}
\label{sec:system_validation}

\subsection{Functional Test: Size}
\label{test:sys_ft_size}
\textbf{Test Question:} How big is each robot agent?\\
\textbf{Operational Procedure:} Measure the physical dimensions of the robot in terms of width, length, and height or in terms of diameter and height. \\
\textbf{Metric:} Numeric value of each length measurement; robot footprint; robot volume.\\
\textbf{Acceptance Criteria:} Must be less than 80 in. x 36 in. x 36 in. \\
\textbf{Requirement(s) Verified:} \nfrref{nfr:size_limit}

\subsection{Functional Test: Weight}
\label{test:sys_ft_weight}
\textbf{Test Question:} How heavy is each robot agent?\\
\textbf{Operational Procedure:} Measure the mass of the robot. \\
\textbf{Metric:} Numeric value of each length measurement; robot footprint; robot volume.\\
\textbf{Acceptance Criteria:} Must be less than 80 in. x 36 in. x 36 in. \\
\textbf{Requirement(s) Verified:} \nfrref{nfr:weight_limit}


Test: do the drive commands respect physical robot capabilities. This relates to whether they robot can drive/draw at the same time w/o marker getting stuck. Also, is there a speed threshold at which drawing quality degrades - involves writing implement, locomotion

Test: does an individual robot draw accurately relative to the commanded motion. This involves making sure line aren't jagged/squiggly, etc. come up with better words for this


Test: does the robot fulfill weight requirements. CHECK REQ SPEC FOR VALUES

Test: is the robot safe, aka does it not have pointy (sharp) edges on the outside

Test/verification: Was robot system price within budget
metric: dollars
acceptance criteria: 2500 dollars


\njnote{Fail: out of bounds - uses localization, locomotion}
\njnote{Fail: robot makes incorrect mark on ground - uses locomotion, localization, writing implement}
\njnote{Fail: collision between robots - uses locomotion, localization}

\subsection{Failure Mode: Out of Bounds}
\label{sec:sys_val_fm_bounds}
\textbf{Description:} This failure describes the situation when any robot agents move beyond the bounds of the drawing surface, as described by the boundary vision markers.\\
\textbf{Cause:} This error can be caused by either poor localization, or poor locomotion. If localization software believes the robot to be somewhere it is not, it may command the robot out of bounds. If the robot motors or wheels are not working properly, it may move out of bounds, despite being given correct motion commands.\\
\textbf{Effect:} The robot moving out of bounds introduces undefined behavior that could result in collision, incorrect drawings, or making marks with the writing tool that are not on the appropriate writing surface.\\
\textbf{Criticality:} This is a critical error, as it results in incorrect localization, drawing marks, and system operation. Entering this failure mode results in system operation ending.\\
\textbf{Safety Hazards:} This failure poses a safety hazard of collision, as robots that exist the bounds may collide with objects or people that it is not expecting.\\

\subsection{Failure Mode: Incorrect Markings}
\label{sec:sys_val_fm_markings}
\textbf{Description:} Incorrect markings are made when a robot agent lowers the writing implement, and makes a mark in a location that does not match with the input drawing. \\
\textbf{Cause:} The cause of incorrect markings could be a result of the writing implement, locomotion, or localization subsystems. The writing implement subsystem may malfunction and lower the tool at an incorrect time. The locomotion subsystem may move the robot incorrectly while the writing implement is lowered, making markings where they are not expected. Localization error can cause markings to be in places that the system thinks are correct, but do not line up with the input drawing.\\
\textbf{Effect:} The effect of this failure is that the output drawing by the system is incorrect.\\
\textbf{Criticality:} Given that the goal of this robot subsystem is to accurately recreate an input drawing, failure to do so is a critical error.\\
\textbf{Safety Hazards:} There are no safety hazards associated with incorrectly marking the drawing surface.\\

\subsection{Failure Mode: Robot Collision}
\label{sec:sys_val_fm_collision}
\textbf{Description:} This failure exists when robot agents collide with any obstacle, including each other or external obstacles.\\
\textbf{Cause:} Robot collision is a result of the locomotion, localization, or work scheduling subsystems failing. Locomotion failure could cause undefined motions by a robot agent, causing it to hit another robot, or move out of bounds (\sref{sec:sys_val_fm_bounds}) and collide with an obstacle. Localization failure could result in incorrect motion commands, resulting in collision with unintended objects. Finally, poor motion planning has potential to require the robots to move into collision with each other.\\
\textbf{Effect:} Robot collision could damage the robot agents, or hurt human observers who are hit.\\
\textbf{Criticality:} This failure is of medium importance. The robots are designed to be safe (\sref{nfr:safe}), and therefore are unlikely to hurt or be significantly damaged by a collision. \\
\textbf{Safety Hazards:} There is a hazard of minor human injury, but as stated above the robots are designed to minimize human injury in the case of collision.\\

