% !TEX root = main.tex

\section{Build Progress}
\label{sec:build_progress}

\rhnote{Add preface}

\subsection{Electromechanical Updates}
\label{sec:electromechanical_progress}
As shown below in \figref{fig:em1}, we have built a physical robot prototype that incorporates chassis, painting mechanism, and locomotion system. Chassis is made of laser patterned acrylic. It is designed to be as compact as possible because smaller robots are less likely to collide with each other during drawing operations. This prototype proves that the chassis’ current cutout sizes have no clearance issue with moving components. 

\begin{figure}[h!]
\centering
\includegraphics[width=0.49\columnwidth]{CAD/full_system_1.jpeg}
\includegraphics[width=0.49\columnwidth]{CAD/full_system_2.jpeg}
\label{fig:em1}
\caption{Prototype Overview, from left view (left) and right view (right)}
\end{figure}

Painting mechanism composes a 3D printed chalk holder and a micro gear motor which is shown in \figref{fig:em2}. The driving motor is mounted to the chassis via an off-the-shelf motor case. When designing the chalk holder, four internal ribs are added inside the holder to securely hold the chalk marker in place while allowing users to easily switch out the marker. A thin cap is added on the bottom of the chalk holder to prevent the chalk marker from sliding out while drawing. One flaw of this design is that the holder’s D-shaft cutout is a little undersized. Therefore, the chalk holder broke while pressing the motor shaft through. This problem will be addressed in the next iteration. 

\begin{figure}[h!]
\centering
\includegraphics[width=0.49\columnwidth]{CAD/painting.jpeg}
\includegraphics[width=0.49\columnwidth]{CAD/chalk_holder.jpeg}
\label{fig:em2}
\caption{Painting Mechanism (left), Chalk Holder CAD (right)}
\end{figure}

\figref{fig:em3} shows the locomotion system. Four Mecanum wheels are oriented in a “X” shape to minimize motor workload. These wheels are connected to driving motors through 3D printed wheel adaptors. These adaptors contain two segments: standard Lego technic axle and D-shaft housing. Like the chalk holder, the D-shaft cutout is a little undersized. Therefore, we had to press fit the motors in. 

\begin{figure}[h!]
\centering
\includegraphics[width=0.32\columnwidth]{CAD/wheels_1.jpeg}
\includegraphics[width=0.32\columnwidth]{CAD/wheels_2.jpeg}
\includegraphics[width=0.32\columnwidth]{CAD/wheel_adapter_1.jpeg}
\label{fig:em3}
\caption{Locomotion System (left), Locomotion System Components (center), Wheel Adaptor CAD (right)}
\end{figure}

Besides mechanical update, motor controller code was also completed. However, we did not get enough time to wire all electronics to this prototype and test the code. This would be the next step of system development. 

Since we have enough left-over budget, we plan to use 80/20 aluminum frames, instead of wood, to construct the camera jig. The jig will be built using components listed in \figref{fig:em4}. We are in the process of testing camera’s optimal height, and will then incorporate that information to the camera jig CAD design. 

\begin{figure}[h!]
\centering
\includegraphics[width=0.98\columnwidth]{CAD/8020.jpeg}
\label{fig:em4}
\caption{80/20 Parts}
\end{figure}

\subsection{Software Update}
\label{sec:software_progress}

\subsubsection{Communication}
\subsubsection{Locomotion}
\subsubsection{Localization}
\subsubsection{Scheduling, Distribution and Planning (SDP)}

In order to our SDP module, we first had to add a few basic UI elements. We laid out a file format for specifying the lines to be drawn and wrote the UI functionality to parse the data in.

Given the data the next step is to distribute the work between the two robots, offline. We will later describe a first pass distribution algorithm along with the UI developed to visualize its results. We will conduct further testing to see if a more advanced algorithm is needed. Luckily, this can be done in parallel with other developments since we have fixed the input and output of the system, allowing us to swap in different distribution tactics. The output of the distributor is a set of vectors that specify the plan for each robot. These vectors will then be handed off to the locomotion module, described above, that will follow each of them in sequence. Therefore, this gives us two next steps: to integrate the planning with the locomotion and to develop a collision avoidance strategy. 

To handle collisions, we will start off with a naive strategy. We define a robot's \textit{boundary} as a fixed radius circle around robot, where the radius exceeds that of the robot to provide cushion. As each robot moves, it will check if the other robot's boundary intersects with its own boundary. If this condition is true, one robot (Bad) will stop execution, allowing the other robot (Blue) to pass until the condition is false. While we believe this method will always prevent collisions, it may not be the most efficient. Therefore we will implement this and test accordingly to check performance. 

For our distributor, we developed a very greedy method. We start Blue the robot at one corner of the drawing area and Bad the robot at the other corner. Our goal is greedily balance their cost, where cost corresponds to the length of the line drawn so far. We initialize both robots with cost zero. From there, we loop over the line count. We pick the robot with the lower cost, defaulting to one in the case of equality. Whichever robot has the lower cost, we pick the line with the closest starting point to the robot's current position. We then calculate the cost as the distance to drive to the line plus the distance to drive to draw out that line. Having updated the robot's position, we continue. 

To illustrate the output of our planner we developed a visualization, shown below in \figref{fig:planner1}. The red path represents Bad the robot and the blue path represents Blue the robot. Solid lines corresponding to drawing lines and hence making a mark on the pavement while dotted lines correspond to purely transit. In \figref{fig:planner1}, Blue the robot starts from the top right corner and transits to draw one line and then return home. In contrast, Bad the robot draws two intersecting, nearby lines.

\begin{figure}[h!]
\centering
\includegraphics[width=0.49\columnwidth]{figs/figure_1.png}
\label{fig:planner1}
\caption{Planning Output}
\end{figure}

In some cases, such as in \figref{fig:planner2}, we see a nice breakdown of work as each robot is responsible for one box. The occurs because at the end of each step, the next closest line is directly adjacent, making the transit distance zero. In \figref{fig:planner3} we see that the greedy approach segments the lines less cleanly and raises the possibility of collision. We note that at every intersection of two robots there is a possibility of collision, but not necessarily if the robots traverse those areas at different times. Additionally the robots could collide outside of intersections since we are not dealing with point robots. As mentioned above, we will continue to explore collision avoidance. 

\begin{figure}[h!]
\centering
\includegraphics[width=0.49\columnwidth]{figs/figure_7.png}
\label{fig:planner2}
\caption{Planning Output}
\end{figure}

\begin{figure}[h!]
\centering
\includegraphics[width=0.49\columnwidth]{figs/figure_4.png}
\label{fig:planner3}
\caption{Planning Output}
\end{figure}