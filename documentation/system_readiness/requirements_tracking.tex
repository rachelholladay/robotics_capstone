% !TEX root = main.tex
\section{Requirements Tracking}
\label{sec:requirements_tracking}

For readbility we provide a summary of our requirements below. 

\begin{center}
 \begin{tabular}{||c c ||}
 \hline
 Requirement & Title \\ [0.5ex]
 \hline\hline
 FR1 & Omnidirectional Movement  \\
 \hline
 FR2 & Autonomous \\
 \hline
 FR3 & Robots Localize Globally and Locally \\
 \hline
 FR4 & Within Bounds \\
 \hline
 FR5 & Insert Writing Tools \\
 \hline
 FR6 & Remove Writing Tools \\
 \hline
 FR7 & Replace Writing Tools \\
 \hline
 FR8 & Coordination \\
 \hline
 FR9 & Drive Control System \\
 \hline
 F10 & Turn on or off writing tool \\
 \hline
 F11 & Input Drawing Plan \\
 \hline
 F12 & Robots Know Progress \\
 \hline
 F13 & Kill Switch \\
 \hline
 F14 & User Interface to Robot \\
 \hline
 NFR1 & Docmentation \\
 \hline
 NFR2 & Error Handling \\
 \hline
 NFR3 & Weight Restriction \\
 \hline
 NFR4 & Size Restriction \\
 \hline
 NFR5 & Efficiency \\
 \hline
 NFR6 & Quality \\
 \hline
 NFR7 & Battery Power \\
 \hline 
 NFR8 & Reliability \\
 \hline 
 NFR9 & Reliable Communication \\
 \hline 
 NFR10 & Budget \\
 \hline 
 NFR11 & Safe \\
 \hline
 NFR12 & Positional Accuracy \\
 \hline
 NFR13 & Rotation Accuracy \\
 \hline
 NFR14 & Tool Switching Duration \\
\hline
\end{tabular}
\end{center}

\subsection{Objectives Tree}
\label{sec:objectives_tree}

We created an objectives tree to better organize our requirements, and prioritize them based on system purposes and goals. After analysis of our requirements, we
formed the objectives tree in \figref{fig:objTreeFull} with the following categories:

\begin{enumerate}
\item Is Safe (\figref{fig:obj_tree_safe})
\item Is Portable (\figref{fig:obj_tree_portable})
\item Drawing Tool is Easy to Operate (\figref{fig:obj_tree_tool})
\item Is Mobile (\figref{fig:obj_tree_mobile})
\item User-Friendly (\figref{fig:obj_tree_user})
\item Performance Guarantees (\figref{fig:obj_tree_performance})
\end{enumerate}

The category for `Is Safe' (\figref{fig:obj_tree_safe}) encompasses requirements for the robot staying within bounds, maintaining reliable communication, existence of a kill switch, and overall safe operation. These requirements breakdown how safe usage of the robot can be achieved, through both system design and user operation.

`Is Portable' (\figref{fig:obj_tree_portable}) specifies system constraints that enable the robots to be able to be transported easily. The battery-powered requirement ensures the robots do not need external power during operation. Weight and size requirements were further categorized into physical constraints, to emphasize the importance of those requirements on portability outside of system operation.

Subtree `Drawing Tool is Easy to Operate' (\figref{fig:obj_tree_tool}) ensures the writing tool is easy to maintain and use both during and before or after system operation. The main subtree describes tool maintenance. Requirements under maintenance include inserting, removing, and replacing the tool, as well as duration requirements for replacing the writing tool. Other requirements in this category relate to having the ability to enage or disengage the writing tool. This requirement is involved with system operation, and ensures the robot can change the tool status so the robots can move regardless of whether it is drawing.

The `Is Mobile' (\figref{fig:obj_tree_mobile}) tree categorizes mobility requirements and constraints for the robot agents. Both positional and rotational accuracy are categorized under their own Accuracy subtree. Other leaves in this subtree ensure the robot agents have their own drive control systems, can localize, and are able to move autonomously in any direction on a 2D plane.

We also chose to separate out requirements that relate to engaging the user and  enable a user-friendly experience. These fall under the `User-Friendly' subtree (\figref{fig:obj_tree_user}). Both documentation and budget requirements were categorized here - these requirements are more likely to be for users interested in adapting or recreating our system. As a result, other requirements were further categorized into a user-interaction subtree. These constraints denote existence of a UI, error handling, and the ability for users to input their drawing plan.

The final categorization, `Performance Guarantees' (\figref{fig:obj_tree_performance}) denotes overall system requirements to ensure the final drawing meets specifications. These requirements include ensuring the robots know their own progress, and coordinate with each other. In addition, requirements specifying system efficiency, reliability, and overall drawing quality fell into this category.

\begin{figure}[!ht]
\centering
\includegraphics[width=0.98\columnwidth]{figs/objectives_tree/objectives_tree_4_2_17.png}
\caption{Full Objectives Tree}
\label{fig:objTreeFull}
\end{figure}

\begin{figure}[!ht]
\centering
\includegraphics[width=0.98\columnwidth]{figs/objectives_tree/objectives_tree_safe.png}
\caption{Objectives Tree: Is Safe Branch}
\label{fig:obj_tree_safe} 
\end{figure}

\begin{figure}[!ht]
\centering
\includegraphics[width=0.98\columnwidth]{figs/objectives_tree/objectives_tree_portable.png}
\caption{Objectives Tree: Is Portable Branch}
\label{fig:obj_tree_portable}
\end{figure}

\begin{figure}[!ht]
\centering
\includegraphics[width=0.98\columnwidth]{figs/objectives_tree/objectives_tree_easy_tool_ops.png}
\caption{Objectives Tree: Drawing Tool is Easy to Operate Branch}
\label{fig:obj_tree_tool}
\end{figure}

\begin{figure}[!ht]
\centering
\includegraphics[width=0.98\columnwidth]{figs/objectives_tree/objectives_tree_mobile.png}
\caption{Objectives Tree; Is Mobile Branch}
\label{fig:obj_tree_mobile}
\end{figure}

\begin{figure}[!ht]
\centering
\includegraphics[width=0.98\columnwidth]{figs/objectives_tree/objectives_tree_user_friendly.png}
\caption{Objectives Tree: User-Friendly Branch}
\label{fig:obj_tree_user}
\end{figure}

\begin{figure}[!ht]
\centering
\includegraphics[width=0.98\columnwidth]{figs/objectives_tree/objectives_tree_performance.png}
\caption{Objectives Tree: Performance Guarantees Branch}
\label{fig:obj_tree_performance}
\end{figure}

\clearpage

\subsection{Requirements Traceability Matrix}
\label{sec:requirements_matrix}

We present our requirements traceability matrix in \figref{fig:req_matrix}. We categorize our requirements into functional and nonfunctional requirements. For each requirement, we classify which subsystem it relates to: writing, communication, locomotion, localization, SDP (scheduling, distribution and planning) and mechanical structure. Each requirement is colored green if initial testing has shown that we achieve the requirement. 

\begin{figure}[h!]
\centering
\includegraphics[width=0.95\textwidth]{figs/requirements_matrix.pdf}
\caption{Requirements Traceability Matrix}
\label{fig:req_matrix}
\end{figure}

\clearpage
