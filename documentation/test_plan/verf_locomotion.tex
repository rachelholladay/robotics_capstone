% !TEX root = system_validation.tex

\subsection{Locomotion}
\label{sec:verification_locomotion}

\subsubsection{Performance Tests}
\label{sec:locomotion_pt}

test of accuracy of positional/rotational accuracy
acceptance - based on requirements spec of pos/rot accuracy
metric - positional/rotational accuracy in distance units

\subsubsection{Functional Tests}
\label{sec:locomotion_ft}

bool - can the robot achieve a desired speed



\subsubsection{Failure Mode: Inaccurate Motion}
\label{sec:locomotion_fm_motion}
\textbf{Description:} This failure mode describes the situation in which a robot agent is unable to accurately follow motion commands. An example of this is if the robot is commanded to move 10 inches, but localization detects it only moving 4.\\
\textbf{Cause:} Inaccurate motion could be a result of slippage of the wheels or failed motor encoders.\\
\textbf{Effects:} Inability to move accurately can cause a user-reported error, which the user can resolve to continue operation. \\
\textbf{Criticality:}  This is a minor failure, as it does not end system operation and can be resolved by the user.\\
\textbf{Safety Hazards:} The only safety hazard is with regard to the drawing surface; slippage could cause minor destruction of the surface.\\

\subsubsection{Failure Mode: Failure to Move Omnidirectionally}
\label{sec:locomotion_fm_omni}
\textbf{Description:} Failure to make omnidirectional movement means a robot agent cannot move in an arbitrary direction on the flat plane represented by the drawing surface.\\
\textbf{Cause:} Similar to \sref{sec:locomotion_fm_motion}, this could be caused by wheels causing slippage in some directions. Alternatively, a broken wheel or motor could have this effect as well.\\
\textbf{Effects:} Failling to move omnidirectionally could result in incorrect drawings - the robot agent can no longer make arbitrarily sharp curves to faithfully recreate the input image. When detected, robot operation should halt and report to the user to be fixed.\\
\textbf{Criticality:}  This is a critical error, as it has a direct relation on the quality of the drawing.\\
\textbf{Safety Hazards:} As with \sref{sec:locomotion_fm_motion}, the only safety hazard is that a broken wheel could damage the drawing surface.\\
