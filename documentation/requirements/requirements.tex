% !TEX root = main.tex

\section{Requirements}
\label{sec:requirements}
We outline our system's requirements, both functional and nonfunctional. A priority number is provided for each requirement. The system requirements are prioritized on a Likert scale from 1 to 7, detailed below. To describe the scale, a 1 is considered a desirable but unnecessary requirement, a 4 is considered a necessary requirement but open to significant changes, and a 7 is considered an uncompromisable requirement for a minimum viable product.

\CheckTable{1}

\subsection{Functional Requirements}
\label{sec:functional_requirements}


\deleted[remark={NJ,V2.2}]{Move in Specified Directions(7): Robots must be able to autonomously move in a commanded direction on a flat plane. Omnidirectional movement is necessary to ensure agents can adequately and efficiently cover the drawing workspace. \added[remark={NJ, V2}]{Positional accuracy must be within a 1 inch radius for every 3 feet of motion. Orientation accuracy must be within 10 degrees for a given rotation.}}

%\begin{functional_requirement}{Omnidirectional Movement}{7}
%\label{fr:omnidirectional}
%\item R\added[remark={NJ,V2.2}]{obots must be able to move instantaneously in any commanded direction on a 2D plane represented by the floor of motion. Omnidirectional movement is necessary to ensure agents can adequately and efficiently cover the drawing workspace, while simultaneously being able to draw lines of varying curvature.}
%\end{functional_requirement}

\begin{functional_requirement}{Omnidirectional Movement}{7}
\label{fr:omnidirectional}
\item R\added[remark={NJ,V2.2}]{obots must be able to move instantaneously in any commanded direction on a 2D plane represented by the floor of motion. Omnidirectional movement is necessary to ensure agents can adequately and efficiently cover the drawing workspace, while simultaneously being able to draw lines of varying curvature.}
\end{functional_requirement}


\njnote{Feedback underlines some words and has an X, not sure what to change}
\begin{functional_requirement}{Autonomous}{6}
\label{fr:autonomous}
\item Given an input image to draw, robot agents must be able to autonomously complete the drawing. This includes the following steps: processing the input, planning and commanding individual agents, and having robots move and draw without external input. We allow for human intervention only in the case of system failure. 
\end{functional_requirement}

\begin{functional_requirement}{Robots Localize Globally and Locally}{7}
\label{fr:localize}
\item All drawing robots can determine their locations and orientations on a global and local scale. Global scale is relative to the localization markers and the specified drawing surface. Local scale is relative to other robot agents. This requirement is necessary so the robot system can coordinate planning together to avoid collisions or an inefficient spread of work. \added[remark={NJ, V2}]{Robot localization should be accurate to within one inch - that is, each robot agent should know its position to within 1 inch of its true location and within 10 degrees of its orientation.}
\end{functional_requirement}

\begin{functional_requirement}{Within Bounds}{3}
\label{fr:in_bounds}
\item While in operation, all robot agents must stay within bounds of the workspace. This is important to minimize external collisions, including possibly unsafe interactions with the world, as well as to ensure localization maintains accuracy while drawing occurs.
\end{functional_requirement}

\deleted[remark={RH, V2}]{
\textbf{Change Writing Tools}
Split into multiple more detailed and specific requirements of insertion, removal, and replacement. Inserting, removing, and replacing writing implements must be convenient and fast.}

\begin{functional_requirement}{Insert Writing Tools}{2}
\label{fr:insert_tool}
\item A\added[remark={NJ, V2}]{ writing implement must be able to be inserted by a user into one robot under 1 minute. New writing implements will need to be inserted whenever existing ones become unusable, or upon first-time setup.}
\end{functional_requirement}

\begin{functional_requirement}{Remove Writing Tools}{2}
\label{fr:remove_tool}
\item .\added[remark={NJ, V2}]{A writing implement must be able to be removed from a single robot by a user under 1 minute. Writing tools will need to be removed when the existing implement becomes unusable.}
\end{functional_requirement}

\begin{functional_requirement}{Replace Writing Tools}{2}
\label{fr:replace_tool}
\item A\added[remark={NJ, V2}]{ user must be able to remove and then insert a writing implement. Following in accordance with Functional Requirements \frref{fr:insert_tool} and \frref{fr:remove_tool}, this process must be done in under 2 minutes. Tool replacement is necessary whenever an existing tool becomes unusable for system operation.}
\end{functional_requirement}

\begin{functional_requirement}{Reliable Communication}{7}
\label{fr:reliable_comm}
\item Robot\added[remark={NJ, V2}]{ agents and any system controllers must have the consistent ability to communicate between each other. This is necessary to ensure accurate and timely planning, and status updates between agents. Connection strength must be such that send information to each other within 1 second.}
\end{functional_requirement}

\begin{functional_requirement}{Drive Control System}{5}
\label{fr:drive_control}
\item Robot agents must have a controller to ensure motions made are accurate. Accurate motions is paramount to ensure an accurate drawing. \added[remark={NJ, V2}]{The drive control system must be in compliance with motion accuracy parameters, as specified in functional requirement, \frref{fr:move_in_direction}.} 
\end{functional_requirement}

\begin{functional_requirement}{Turn on or off writing tool}{1}
\label{fr:on_tool}
\item All robot agents need to be able to enable or disable use of the writing implement. The robot must also have the ability to move regardless of the state of the writing implement. Not all drawings are contiguous lines, and as such the robots must be able to disable the writing tool to move to a new drawing location.
\end{functional_requirement}

\begin{functional_requirement}{Input Drawing Plan}{6}
\label{fr:input_plan}
\item The main system controller must be able to receive an input that allows it to command the robot agents to draw an appropriate image. The system must be able to parse the input into a state usable for robot planning and control in order to draw the input. 
\end{functional_requirement}

\begin{functional_requirement}{Robots Know Progress}{2}
\label{fr:know_progress}
\item All robot agents are required to understand how much of the drawing each one has completed, and which sections are left to be drawn. This is necessary to ensure an equal spread of workload across all robot agents. \added[remark={NJ, V2}]{The ability to communicate progress can be guaranteed by consistent inter-robot communication, as defined by \frref{fr:reliable_comm}.}
\end{functional_requirement}

\begin{functional_requirement}{Kill Switch}{3}
\label{fr:kill_switch}
\item Human bystanders must be able to end all system operation instantaneously with a kill switch or power button. This is necessary to ensure that the system can be shut down in case of an unsafe error or problem. 
\end{functional_requirement}

\begin{functional_requirement}{User Interface to robot}{1}
\label{fr:user_interface}
\item An intuitive and useful user experience is necessary for efficient usage of the system. Having a simple to operate system also reduces the likelihood of user error during the input or operation stage. For industrial or commercial applications, accessibility becomes important as well. \added[remark={NJ, V2}]{A quality user interface can be tested by running a user study or survey with potential users.}
\end{functional_requirement}

\deleted[remark={RH, V2}]{
\textbf{Documentation} \\
Documentation of the design process, software, and hardware implementation is important for debugging, recreation, and general understanding of this project.}

\begin{functional_requirement}{Battery Power}{4}
\label{fr:battery_power}
\item I\added[remark={NJ,V2.2}]{ndividual robot agents must be able to run continuously for a minimum of 30 minutes before needing battery charging or replacement. This will provide enough time to complete a minimum of a single drawing.}
\end{functional_requirement}


\subsection{Non-Functional Requirements}
\label{sec:nonfunctional_requirements}

\begin{nonfunctional_requirement}{Documentation}{3}
\item Documentation\added[remark={NJ, V2}]{ of the design process, software, and hardware implementation is important for debugging, recreation, and general understanding of this project.}
\end{nonfunctional_requirement}

\begin{nonfunctional_requirement}{Error Handling}{6}
\item The\added[remark={NJ, V2}]{ system must be able to handle errors appropriately. This includes problems that arise from localization, locomotion, planning, or using the writing tool. The system must be able to determine, based on conditions of the failure, whether to halt all operation or to continue without using broken subsystems. Errors can occur inside of any of the use cases listed in \sref{sec:use_cases}.}
\end{nonfunctional_requirement}

\deleted[remark={NJ,V2.2}]{Portable(4): Individual robot agents must be easy to move between locations. Portability can be measured by physical dimensions and weight. Each robot must fit through a standard doorway (defined as 80 in. x 36 in. \cite{homedepotdoor}), and be under 50 pounds.}

\begin{nonfunctional_requirement}{Weight Restriction}{4}
\label{nfr:weight_limit}
\item I\added[remark={NJ,V2.2}]{ndividual robot agent weight must be under 50 pounds. This is one step in ensuring portability of the individual agents.}
\end{nonfunctional_requirement}

\begin{nonfunctional_requirement}{Size Restriction}{5}
\label{nfr:size_limit}
\item I\added[remark={NJ,V2.2}]{ndividual robot agent size must be able to fit within a standard doorway. This is defined as 80 in. x 36 in. \cite{homedepotdoor}.}
\end{nonfunctional_requirement}

\begin{nonfunctional_requirement}{Efficiency}{2}
\item Robot system must be efficient and complete drawing tasks quickly. Timeliness can be measured by the amount of time taken to complete drawings. Robot planning must also exist in such a manner that evenly splits the workload among all robot agents working in the system. Therefore, a speedup of a maximum of 2x the rate is expected when using two robots as opposed to one. \added[remark={NJ, V2}]{The system should perform as fast or faster than a human attempting the same drawing task.}
\end{nonfunctional_requirement}

\begin{nonfunctional_requirement}{Quality}{4}
\item The drawing that results from the robotic system must closely match the input image. Quality can be qualitatively measured by visual comparison, or using software via a number of image difference metrics.
\end{nonfunctional_requirement}

\begin{nonfunctional_requirement}{Mobile App}{1}
\item User interface and experience will be done using a mobile application. This application will allow users to remotely input images for drawing, enable, disable, and pause the robot system's operation. The app will also be able to track and display current progress.
\end{nonfunctional_requirement}

\begin{nonfunctional_requirement}{Reliability}{6}
\item Robot system must be robust, and be resilient to breaking down or failing to operate properly. Reliability can be measured by percent uptime relative to total time spent in use.
\end{nonfunctional_requirement}

\deleted[remark={NJ,V2.2}]{Battery Life(2): Individual robot agents must have a battery life capable of completing a minimum of a single drawing. Battery life can be measured by duration for which robot agents are in use before needing battery recharging or replacement.}

\begin{nonfunctional_requirement}{Coordination}{4}
\item The key to a multi-robot agent system is to reduce the individual workload, meaning having coordinated and efficient work is vital. Robots must be able to work together to minimize overlap in the drawings, and to avoid duplicating work. A two-robot system that coordinates must perform faster than a single robot performing the same task.
\end{nonfunctional_requirement}

\begin{nonfunctional_requirement}{Budget}{7}
\label{nfr:budget}
\item Design and implementation of this robotic system is limited by budget, which must be strictly adhered to. \added[remark={NJ, V2}]{The budget is limited \$2500.}
\end{nonfunctional_requirement}

\begin{nonfunctional_requirement}{Safe}{4}
\label{nfr:safe}
\item Robots must maintain safety with respect to each other and the external world at all times. This requires that all robot agents avoid collisions. Robot agents must also have an enforced maximum speed limit to avoid damage to themselves or human bystanders in case of collision. 
\end{nonfunctional_requirement}

\begin{nonfunctional_requirement}{Positional Accuracy}{7}
\label{nfr:pos_accuracy}
\item R\added[remark={NJ,V2.2}]{obots must be able to move with a positional accuracy of a 1 inch radius for every 3 feet of motion. Tight positional accuracy is vital to ensuring the robots can accurately complete the drawing.}
\end{nonfunctional_requirement}

\begin{nonfunctional_requirement}{Rotational Accuracy}{7}
\label{nfr:rot_accuracy}
\item R\added[remark={NJ,V2.2}]{obots must be able to turn with a rotational accuracy of within 10 degrees per 90 degrees turned. Similar to \sref{nfr:rot_accuracy}, rotational accuracy is necessary for the robots to accurately complete the drawing.}
\end{nonfunctional_requirement}

\begin{nonfunctional_requirement}{Tool Switching Duration}{2}
\label{nfr:tool_switch_duration}
\item I\added[remark={NJ,V2.2}]{ndividual robots must be able to enable or disable the writing tool within 10 seconds.}
\end{nonfunctional_requirement}


