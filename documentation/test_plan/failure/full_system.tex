% !TEX root = main.tex

\subsection{Full System}
\label{failure:full_system}

\subsubsection{Out of Bounds}
\label{sec:sys_val_fm_bounds}
\added[remark={RH, V2}]{\textbf{Description:} This failure describes the situation when any robot agents move beyond the bounds of the drawing surface, as described by the boundary vision markers.}\\
\added[remark={RH, V2}]{\textbf{Cause:} This error can be caused by either poor localization, or poor locomotion. If localization software believes the robot to be somewhere it is not, it may command the robot out of bounds. If the robot motors or wheels are not working properly, it may move out of bounds, despite being given correct motion commands.}\\
\added[remark={RH, V2}]{\textbf{Effect:} The robot moving out of bounds introduces undefined behavior that could result in collision, incorrect drawings, or making marks with the writing tool that are not on the appropriate writing surface.}\\
\added[remark={RH, V2}]{\textbf{Criticality:} This is a critical error, as it results in incorrect localization, drawing marks, and system operation. Entering this failure mode results in system operation ending.}\\
\added[remark={RH, V2}]{\textbf{Safety Hazards:} This failure poses a safety hazard of collision, as robots that exist the bounds may collide with objects or people that it is not expecting.}\\
\added[remark={DZ, V2}]{\textbf{Mitigation:} Robust localization system  and driving mechanism designs can help mitigate this failure mode.}\\
\added[remark={RH, V2}]{\textbf{Failure Tree:} See \figref{fig:out_of_bounds_failure}}

\begin{figure}
 \centering
 \includegraphics[width=0.65\columnwidth]{figs/fault_table_bounds.png}
 \caption{Fault Tree of the Failure Mode of the Robot going out of bounds.}
 \label{fig:out_of_bounds_failure}
\end{figure}
\clearpage

\subsubsection{Incorrect Markings}
\label{sec:sys_val_fm_markings}
\added[remark={RH, V2}]{\textbf{Description:} Incorrect markings are made when a robot agent lowers the writing implement, and makes a mark in a location that does not match with the input drawing. }\\
\added[remark={RH, V2}]{\textbf{Cause:} The cause of incorrect markings could be a result of the writing implement, locomotion, or localization subsystems. The writing implement subsystem may malfunction and lower the tool at an incorrect time. The locomotion subsystem may move the robot incorrectly while the writing implement is lowered, making markings where they are not expected. Localization error can cause markings to be in places that the system thinks are correct, but do not line up with the input drawing.}\\
\added[remark={RH, V2}]{\textbf{Effect:} The effect of this failure is that the output drawing by the system is incorrect.}\\
\added[remark={RH, V2}]{\textbf{Criticality:} Given that the goal of this robot subsystem is to accurately recreate an input drawing, failure to do so is a critical error.}\\
\added[remark={RH, V2}]{\textbf{Safety Hazards:} There are no safety hazards associated with incorrectly marking the drawing surface.}\\
\added[remark={DZ, V2}]{\textbf{Mitigation:} A fail-safe writing implement design and robust localization/driving mechanism designs can help mitigate this failure mode.}


\subsubsection{Robot Collision}
\label{sec:sys_val_fm_collision}
\added[remark={RH, V2}]{\textbf{Description:} This failure exists when robot agents collide with any obstacle, including each other or external obstacles.}\\
\added[remark={RH, V2}]{\textbf{Cause:} Robot collision is a result of the locomotion, localization, or work scheduling subsystems failing. Locomotion failure could cause undefined motions by a robot agent, causing it to hit another robot, or move out of bounds (\sref{sec:sys_val_fm_bounds}) and collide with an obstacle. Localization failure could result in incorrect motion commands, resulting in collision with unintended objects. Finally, poor motion planning has potential to require the robots to move into collision with each other.}\\
\added[remark={RH, V2}]{\textbf{Effect:} Robot collision could damage the robot agents, or hurt human observers who are hit.}\\
\added[remark={RH, V2}]{\textbf{Criticality:} This failure is of medium importance. The robots are designed to be safe (\nfrref{nfr:safe}), and therefore are unlikely to hurt or be significantly damaged by a collision.} \\
\added[remark={RH, V2}]{\textbf{Safety Hazards:} There is a hazard of minor human injury, but as stated above the robots are designed to minimize human injury in the case of collision.}\\
\added[remark={DZ, V2}]{\textbf{Mitigation:} Robust localization and driving mechanism designs, as well as short-rage collision sensors can help mitigate this failure mode.}

\subsubsection{Intruder Collision}
\label{sec:sys_val_fm_intruder}
\added[remark={DZ, V2}]{\textbf{Description:} Robot agents run the risk of colliding with intruders when humans or other objects unexpectedly move into the robot's workspace during operation.}\\
\added[remark={DZ, V2}]{\textbf{Cause:} Intruder collision can occur when humans and other objects are allowed to enter the workspace.}\\
\added[remark={DZ, V2}]{\textbf{Effect:} Intruding objects prevent robots from moving in their intended directions, severely affecting effectiveness of operation.}\\
\added[remark={DZ, V2}]{\textbf{Criticality:} This failure is of high importance. If an intruder occupies the same area as a part of the drawing, it will be impossible for the system to complete the task.}\\
\added[remark={DZ, V2}]{\textbf{Safety Hazards:} There is a hazard for minor human injury if very young humans are caught in the wheels or motors.}\\
\added[remark={DZ, V2}]{\textbf{Mitigation:} Tight security around the workspace, whether through physical barriers or signs, can help mitigate this failure mode.}\\

\subsubsection{Finger Jam}
\label{sec:sys_val_fm_finger}
\added[remark={DZ, V2}]{\textbf{Description:} This failure occurs when a human user's fingers are jammed in the motors for chalk lifting or locomotion.}\\
\added[remark={DZ, V2}]{\textbf{Cause:} Finger jamming can occur as result of improper operation of the robot, or if untrained users are allowed into the workspace during operation.}\\
\added[remark={DZ, V2}]{\textbf{Effect:} The motors could damage jammed fingers, causing injury to humans. The motors themselves would likely also be damaged.}\\
\added[remark={DZ, V2}]{\textbf{Criticality:} This failure is of high importance. It directly affects human safety and the operation of the entire system.}\\
\added[remark={DZ, V2}]{\textbf{Safety Hazards:} There is a hazard for severe human injury in the worst case.}\\
\added[remark={DZ, V2}]{\textbf{Mitigation:} Strict procedures and instructions for handling the robot can help mitigate this failure mode. Additionally, safety housings over moving parts will also decrease the likelihood of the failure mode occuring.}\\