% !TEX root = main.tex

\section{Subsystem Descriptions}
\label{sec:subsystem_descriptions}

\rhnote{Rubric: All subsystem descriptions provide a clear idea of each subsystem and any critical component is identified.}

\rhnote{Some subsystems might need their own diagram}


\subsection{Writing Implement Actuator}
\label{sec:subsystem_writing_implement_actuator}
\textbf{Description:} The writing implement actuator deposits writing material onto the working surface. It also manages reloading of the writing implement and ensures that the implement is properly secured. In the event of writing material depletion, its sensors will detect the occurence and alert the communication module (\sref{sec:subsystem_communication}).\\
\textbf{Critical Components:} Writing implement holder, depositor, writing material levels sensor, reloading mechanism.\\

\subsection{Locomotion}
\label{sec:subsystem_locomotion}
drives individual robot. want to go sideways. have enough clearence for writing tool. based on trade studies need wheels.
\textbf{Description:} The robot's locomotion system propels each individual robot across the working surface. As the robots use holonomic locomotion systems, they are able to move in any direction, ensuring their ability to execute intricate designs. They also provide enough clearance for the writing implement to work effectively.\\
\textbf{Critical Components:} Wheels, motors.\\


\subsection{Localization}
\label{sec:subsystem_localization}
\textbf{Description:} The localization system uses input from the sensor module to accurately determine an individual robot's position and orientation. It then communicates the information to the scheduling module (\sref{sec:subsystem_planner}). 
\textbf{Critical Components:} Localization algorithm

\subsection{Input Image Scanner}
\label{sec:subsystem_input_scanner}
\textbf{Description:} The image scanner takes a user-provided image or generates one, and transfers it to the image processing module (\sref{subsystem_image_processor}). The image scanner, being a module requiring user interaction, will also incorporate a front-end user interface.\\
\textbf{Critical Components:} User interface, image sensor.\\

\subsection{Image Processor}
\label{sec:subsystem_image_processor}
needs to process image. take image scanned by user and parse it into input regonizable by work scheduling, distribution and planning
\textbf{Description:} The image processor takes input from the image scanner (\sref{sec:subsystem_input_scanner}) and produces information that is recognizable by the planning module (\sref{subsystem_planner}). 
\textbf{Critical Components:} Image processing algorithm.

\subsection{Work Scheduling, Distribution and Planning}
\label{sec:subsystem_planner}
see software architecture. couple small senteces.
\textbf{Description:} Given the output from the image processor (\sref{subsystem_image_processor}), and parameters such as the number of robots and the size of the workspace, the module determines the work that will be distributed to each robot. See software architecture (\sref{sec:software_architecture}) for more information.\\
\textbf{Critical Components:} Scheduling algorithm.\\

\subsection{Communication}
\label{sec:subsystem_communication}
antenna fit within requirements. have certain range. allow multi agent communication. want to be fairly speedy and realiable. possible communication protocols: bluetooth, wifi, serial communication (like mav link),
\textbf{Description:} The communication module is the link between the offboard system and the individual robots. To facilitate real-time changes in the working schedule, communication will be speedy and reliable. \\
\textbf{Critical Components:} Antennae, wireless communication protocol.\\

\subsection{User Interface}
\label{sec:subsystem_ui}
\todo{Insert flow chart for box \"specify drawing settings\" from concept operation}
\textbf{Description:} The user interface provides a unified system for user input. This is the system with which the user specifies the input image and monitors the robots' progress. There will also be a system-wide kill switch in case of emergencies.\\
\textbf{Critical Components:} Screen, input device.\\

\subsection{Sensor Module}
\label{sec:subsystem_sensors}
\textbf{Description:} The sensor module gathers data from environmental markers (\sref{subsystem_markers}). This data is used by the localization module (\sref{subsystem_localization}) to determine an individual robot's position and orientation in the workspace.\\
\textbf{Critical Components:} Beacon sensors.\\

\subsection{Power System}
\label{sec:subsystem_power}
Needs to onboard power. probably battery. Battery needs to small and light enough to satisfy the size and weight requirement.
\textbf{Description:} The power system supplies power to the rest of the system. Each robot has an onboard battery that is small and light enough to satisfy the size and weight requirement, but also provides enough uptime to last a entire drawing session.\\
\textbf{Critical Components:} Battery, voltage regulator modules.\\

\subsection{Environmental Markers}
\label{sec:subsystem_markers}
Fixed. Put markers on all 4 sides. Calibrated in initialization. Reference trade study
\textbf{Description:} The environmental markers must be set up by the user before the system can begin drawing. They provide the data needed by the localization subsystem (\sref{subsystem_localization}) to determine the robot's position and orientation in the workspace. They will be mounted high enough as to be visible by every robot in the workspace at all times.
\textbf{Critical Components:} Beacons/markers, elevated stands.