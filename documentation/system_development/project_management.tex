% !TEX root = main.tex
\subsection{Work Breakdown Schedule}
\label{sec:wbs}

In this section, we present the Work Breakdown Schedule for the project.

\begin{figure}[h!]
\centering
\includegraphics[width=\textwidth]{wbs_schedule/wbs_1_31_17.png}
\caption{Full WBS for the project}
\label{fig:full-wbs}
\end{figure}

\begin{figure}[h!]
\centering
\includegraphics[width=0.8\columnwidth]{wbs_schedule/wbs_electromechanical.png}
\caption{Electromechanical WBS section}
\label{fig:full-wbs}
\end{figure}
\begin{figure}[h!]
\centering
\includegraphics[width=0.9\columnwidth]{wbs_schedule/wbs_software.png}
\caption{Software WBS section}
\label{fig:full-wbs}
\end{figure}
\begin{figure}[h!]
\centering
\includegraphics[width=0.9\columnwidth]{wbs_schedule/wbs_integration.png}
\caption{Integration WBS section}
\label{fig:full-wbs}
\end{figure}

\clearpage

The WBS dictionary entries include more information on each of the work elements of the project. Information such as estimates for the amount of time each task will take and their dependencies will help us adhere to our schedule, while determining the owner of each task will improve tractability of the workflow.

\begin{figure}[h!]
\centering
\includegraphics[width=0.98\columnwidth]{wbs_schedule/wbs_dict_hw1.PNG}
\label{fig:hw1}
\end{figure}
\begin{figure}[h!]
\centering
\includegraphics[width=0.98\columnwidth]{wbs_schedule/wbs_dict_hw2.PNG}
\label{fig:hw2}
\end{figure}
\begin{figure}[h!]
\centering
\includegraphics[width=0.98\columnwidth]{wbs_schedule/wbs_dict_hw3.PNG}
\label{fig:hw3}
\end{figure}
\begin{figure}[h!]
\centering
\includegraphics[width=0.98\columnwidth]{wbs_schedule/wbs_dict_hw4.PNG}
\label{fig:hw4}
\end{figure}
\begin{figure}[h!]
\centering
\includegraphics[width=0.98\columnwidth]{wbs_schedule/wbs_dict_hw5.PNG}
\label{fig:hw5}
\end{figure}
\begin{figure}[h!]
\centering
\includegraphics[width=0.98\columnwidth]{wbs_schedule/wbs_dict_hw6.PNG}
\label{fig:hw6}
\end{figure}
\begin{figure}[h!]
\centering
\includegraphics[width=0.98\columnwidth]{wbs_schedule/wbs_dict_hw7.PNG}
\label{fig:hw7}
\end{figure}
\begin{figure}[h!]
\centering
\includegraphics[width=0.98\columnwidth]{wbs_schedule/wbs_dict_sw1.PNG}
\label{fig:hw1}
\end{figure}
\begin{figure}[h!]
\centering
\includegraphics[width=0.98\columnwidth]{wbs_schedule/wbs_dict_sw2.PNG}
\label{fig:hw2}
\end{figure}
\begin{figure}[h!]
\centering
\includegraphics[width=0.98\columnwidth]{wbs_schedule/wbs_dict_sw3.PNG}
\label{fig:hw3}
\end{figure}
\begin{figure}[h!]
\centering
\includegraphics[width=0.98\columnwidth]{wbs_schedule/wbs_dict_sw4.PNG}
\label{fig:hw4}
\end{figure}
\begin{figure}[h!]
\centering
\includegraphics[width=0.98\columnwidth]{wbs_schedule/wbs_dict_sw5.PNG}
\label{fig:hw5}
\end{figure}
\begin{figure}[h!]
\centering
\includegraphics[width=0.98\columnwidth]{wbs_schedule/wbs_dict_sw6.PNG}
\label{fig:hw6}
\end{figure}
\begin{figure}[h!]
\centering
\includegraphics[width=0.98\columnwidth]{wbs_schedule/wbs_dict_sw7.PNG}
\label{fig:hw7}
\end{figure}
\begin{figure}[h!]
\centering
\includegraphics[width=0.98\columnwidth]{wbs_schedule/wbs_dict_int1.PNG}
\label{fig:int1}
\end{figure}
\begin{figure}[h!]
\centering
\includegraphics[width=0.98\columnwidth]{wbs_schedule/wbs_dict_int2.PNG}
\label{fig:int1}
\end{figure}
\begin{figure}[h!]
\centering
\includegraphics[width=0.98\columnwidth]{wbs_schedule/wbs_dict_int3.PNG}
\label{fig:int1}
\end{figure}

\clearpage

\section{Schedule}
\label{sec:schedule}

\figref{fig:gantt_3_7} shows an updated Gantt chart of our system development progress. Sections colored in gray represent completion status of various tasks. Overall, we are on schedule, with few changes. The UI subsystem has been moved forward to the end of spring break, as our team has time to work on it then. In addition, the UI system has been developed in conjunction with the planning system, and as such is already partially completed.

Electromechanical progress remains on schedule, with slight delays and adjustments due to parts taking longer than anticipated to deliver. The chassis is completed, and is currently in testing for potential issues such as vibration causing unstable writing motions. The writing implement is mostly completed, but due to a broken part is currently being refabricated. The camera rig is slightly behind, and we have decided to alter the material used to design it. Instead of wood, we switched to using 80/20, which are designed for simple construction and adjustment.

Software progress remains on schedule. The scheduling subsystem is currently operational, with some integration left. The communication subsystem has been completed, localization is being finished, and locomotion is in complete, but being updated to account for design changes.

\begin{figure}[h!]
\centering
\includegraphics[width=0.49\columnwidth]{figs/gantt_chart_3_1_17.png}
\caption{Prototype Overview, from left view (left) and right view (right)}
\label{fig:gantt_3_7}
\end{figure}
