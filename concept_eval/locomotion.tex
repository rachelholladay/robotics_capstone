% !TEX root = main.tex

\section{Locomotion}
\label{sec:locomotion}
\textbf{Description:} The robot's locomotion system propels each individual robot across the working surface. The robot uses a holonomic locomotion system, it can move in any direction, which allows the robots to execute intricate designs. \added[remark={DZ, V2}]{Mechanum wheels will allow the robots that flexibility, while at the same time preserving stability and ease of control.} They also provide enough clearance for the writing implement to work effectively. A driving system that allows for multidirectional movement satisfies motion-related requirements (Requirements Specification, 5.1, FR7). Additionally, the locomotion system contains encoders whose data is used by the localization module (\sref{sec:localization}). \\

\noindent
\textbf{Critical Components:} Wheels or treads, motors and encoders.

\subsection{Use Cases}
\subsubsection{Locomote}
\textbf{Description:} \added[remark={DZ, V2}]{The robot can use its array of motors and wheels to propel itself across the flat working surface. It can make arbitrarily sharp turns and acute curves in order to allow input drawings that incorporate such components. }

\subsection{Trade Study}
\label{sec:trade_locomotion}
One of our functional requirements with the highest priority (FR1) is to be able to move in specified directions with a high degree of accuracy. Additionally, a medium high priority functional requirement (FR 9) is to have a drive control system that enables this movement. Therefore, it is clear from our requirements that locomotion is critical to our robot's performance. When considering locomotion options there are three large categories: aerial, legged and wheeled.

Our drawing occurs on the 2D plane. Therefore, any flight based system would have to constrain one translational degree of freedom and two rotational degrees of freedom. Doing so would require precise control algorithm and large power input. Practically speaking, this overcomplicates the problem for little benefit, and as such rules out airborne locomotion systems.

Therefore, locomotion options are limited to a legged or wheeled system. While a legged system would have the ability to traverse uneven terrain, such capability is not necessary for this project due to the assumption (A3) that the drawing surface is flat and homogenous. Compared to wheeled systems, legged systems usually involve more complicated control algorithms and electrical and mechanical components. In addition, legged systems often fail to ensure stability in the robot body, which for this system holds the writing implement. Considering system complexity and locomotion stability, it can be concluded a wheeled system is best.

In continuing to prioritize functional requirement (FR1), we investigate other possible wheeled drive systems. Four different drive systems are evaluated: differential drive, Ackerman steering, four-wheel steering, and holonomic drive. \added[remark={YJ, V2}]{Each of the four drive systems are evaluated based on mobility, motion accuracy and amount of damage done to the drawing surface.

Differential drive systems typically consist of two independently driven wheels and a non-actuated wheel. Such a drive system has good mobility due to its minimum number of driving wheels. However, the non-actuated wheel often creates unwanted resistance while turning, which both compromises the drawing accuracy and damages the drawing surface. Occasionally, the non-driven wheel can fall into a singularity and ruin drawing quality.

Next is Ackerman steering, where the back pair of wheels is fixed in orientation but the front pair can pivot. This drive system does not suffer from singularities and does not damage drawing surfaces. However, turning with ackerman steering often requires a large turning radius dependant on the wheel base, which makes sharp corners difficult to draw. Despite a high accuracy of motion and minimal damage to the drawing surface, Ackerman steering has a low mobility and is not feasible.

Another option is four-wheel steering, which is similar to Ackerman steering but allows the back wheels to pivot as well. Even though such drive system can achieve in-place turn, doing so often requires the wheels to pivot in place, which can potentially damage the drawing surface. Hence, four-wheel steering ranks high in mobility and motion accuracy, but low in damage to drawing surfaces.

Holonomic drive involves independent control of four omnidirectional wheels. Such drive system can easily turn in place and does not cause any potential damage to drawing surfaces. The drive system's motion accuracy depends on wheel choice. Holonomic drive is often achieved with either omniwheels or mecanum wheels. Omniwheel does not resist any force in its lateral direction, resulting in low motion accuracy due to slippage. Mecanum wheels, on the other hand, resist sideways disturbances and ensure straight motion due to the wheel's obliquely oriented rollers.

In conclusion, the system's locomotion system will be a wheeled holonomic drive system with mecanum wheels, since such combination best satisfies system requirements and performs best in criteria concerning mobility, motion accuracy, and amount of damage to drawing surfaces.}

\subsection{Artistic Sketch}

\begin{figure}[!ht]
 \centering
  \includegraphics[width=0.95\columnwidth]{sketches/Locomotion.jpg} 
	\caption{Locomotion Sketch}
 \label{fig:locomotion_sketch}
\end{figure}
\clearpage

\subsection{Requirements Fulfilled}
\added[remark={RH, V2}]{By using mecanum wheels, we can achieve omnidirectional movement (FR1) and reliable control (FR9). Our emergency stop (FR13) and error handling (NFR2) will be integrated into the drive system through the communication network \sref{sec:communication}. Our mecanum wheels have been specificed to stay within our weight (NFR3), size (NFR4) and budget (NFR10) limits. The control of the mecanums will allow us to be efficient (NFR5), due to their omnidirectional motion and speed, while maintaining quality (NFR6), reliability (NFR8) and safety (NFR11). Additionally the omnidirectional control wil allow us positional (NFR12) and rotational (NFR13) accuracy.}
