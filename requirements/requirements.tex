% !TEX root = main.tex

\section{Requirements}
\label{sec:requirements}
We outline our systems requirements, both functional and nonfunctional. For each requirement we provide a number, for each of reference, a short description and a longer explanation. We prioritize our system requirements on a Likert scale from 1 to 7, detailed below. 

\CheckTable{1}

\subsection{Functional Requirements}
\label{sec:functional_requirements}

\begin{functional_requirement}{Move in 4 Directions}{1}
\item Fill in more details 
\end{functional_requirement}

\begin{functional_requirement}{Autonomous}{1}
\item Fill in more details 
\end{functional_requirement}

\begin{functional_requirement}{Robots Localize Globally and Locally}{1}
\item Fill in more details 
\end{functional_requirement}

\begin{functional_requirement}{Safe}{1}
\item Doesnt run into things. No dangerous external parts
\end{functional_requirement}

\begin{functional_requirement}{Within Bounds}{1}
\item Stay within bounds of drawing
\end{functional_requirement}

\begin{functional_requirement}{Change Tools}{1}
\item Easy to swap out tools
\end{functional_requirement}

\begin{functional_requirement}{Drive Control System}{1}
\item Details
\end{functional_requirement}

\begin{functional_requirement}{Turn on or off writing tool}{1}
\item So lift pencil up and down while moving. That way doesn't have to be continuous lines
\end{functional_requirement}

\begin{functional_requirement}{Input drawing Plan}{1}
\item how well can you give it commands
\end{functional_requirement}

\begin{functional_requirement}{Robots Know Progress}{1}
\item Keep track of how much you have drawn
\end{functional_requirement}

\begin{functional_requirement}{Kill Switch}{1}
\item immediately powers off robots for safety
\end{functional_requirement}

\begin{functional_requirement}{User Interface to robot}{1}
\item how it gets controlled
\end{functional_requirement}

\begin{functional_requirement}{Be In Budget}{1}
\item how it gets controlled
\end{functional_requirement}

\begin{functional_requirement}{Documentation}{1}
\item Keep code and design documentation
\end{functional_requirement}


\subsection{Non-Functional Requirements}
\label{sec:nonfunctional_requirements}

\begin{nonfunctional_requirement}{Portable}{1}
\item Small, can be carried, easy to move around. Weight less than 50 pounds. Size: bigger than 2 foot cube
\end{nonfunctional_requirement}

\begin{nonfunctional_requirement}{Completes Task in Timely manner}{1}
\item Details
\end{nonfunctional_requirement}

\begin{nonfunctional_requirement}{Quality}{1}
\item Matches input well. 
\end{nonfunctional_requirement}

\begin{nonfunctional_requirement}{Mobile App}{1}
\item Neil go nuts on your bullshit
\end{nonfunctional_requirement}

\begin{nonfunctional_requirement}{Reliability}{1}
\item Percent up time
\end{nonfunctional_requirement}

\begin{nonfunctional_requirement}{Battery Life}{1}
\item Needs to last
\end{nonfunctional_requirement}

\begin{nonfunctional_requirement}{Fault Tolerance}{1}
\item Needs to last
\end{nonfunctional_requirement}

\begin{nonfunctional_requirement}{Coordination}{1}
\item dont duplicate work or overlap
\end{nonfunctional_requirement}

\begin{nonfunctional_requirement}{Efficiency}{1}
\item split up work evenly
\end{nonfunctional_requirement}
