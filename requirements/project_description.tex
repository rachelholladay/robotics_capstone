% !TEX root = main.tex

\section{Project Description}
\label{sec:project_description}
We plan to build a multi-agent autonomous robotic drawing system. Our goals, motivation, \added[remark={New Section}]{definitions,} scope and assumptions are provided below.

\subsection{Product Goal}
\label{sec:project_goal}
The goal of our project is to use robotics to bring people's ideas on paper into a reality. 
We aim to build a system that takes an image as input and reproduces that image on some surfaces, such as ground, poster, or pavement.
This direct interaction with the physical world drives much of our design, as well as the need to use a dispensable writing tool, such as chalk or marker. 

By using multiple robots, we hope to efficiently decompose a possibly large task into smaller, independent pieces which could be completed in parallel.
In doing so we can also explore various coordination and scheduling strategies that naturally arise in a multi-robot scenario. 

\subsection{Definitions}
\label{sec:definitions}

\added[remark={Definition Section}]{
\underline{\textbf{System.}}
The system includes all robots, user interface and any accessories, such as localization markers. \\
\underline{\textbf{Robot.}}
A single robot agent within the system.  
}

\subsection{Motivation}
\label{sec:motivation}

Traditionally robots have been pitched as excelling in the three D's: dull, dirty and dangerous. Our project primarily focuses on automating a dull task. The United States has nearly 47,000 miles of interstate highway and more than 5,000 airports with paved runways \cite{buildfuture}. Each of these pieces of infrastructure in delineated and marked with drawn lines, which, when worn, must be repainted. These tasks are time consuming, expensive and, in many occasions, dirty, for example painting bicycle lanes would cost on average 133,170 dollars per mile \cite{bicyclist}.

By enabling robotic automation in painting these lines, we can improve the quality of our infrastructure while saving time and money in the long term. We can expand past transportation infrastructure to sporting arenas. Sports such as baseball, football or American football have fields with markings that must be regularly maintained to insure fair game play. 

While this robot can serve a very functional purpose, it is not without its playful side. The robot can potentially be used as a children's toy for bringing the imagination of drawings to life. 

\added[remark={Clarify scope.}]{While a fully operational system would ideally be able to cover all of the above motivational examples, we do not expect our system to. We discuss the expected scope of our project in \sref{sec:project_scope}}.

\subsection{Product Scope}
\label{sec:project_scope}
While our \deleted[remark={Confusing Wording}]{robotic} system has a great deal of potential, we want to insure that we can reasonably achieve testable goals. Therefore, in this section we will discuss robotic function and scenarios which are out of scope of this project. 

We describe as system has being a multi-agent system. Based on our task, the number of robots in the system could scale up immeasurably. Due to time, cost and planning constraints, we begin with a multi-agent system of two \added[remark={Add Clarification}]{homogenous} robots. 

In our motivation and goal, we mention large scale applications such as airport runways and stadiums. However, for ease of testing we intend to primarily target this early version to smaller scale projects, such as drawing a design on a large, indoor poster. 

Inherently, there is little that prevents our proposed project from scaling in this dimension aside from increasing the durability of our system. This level of durability, to functional well outdoors, is considering out of scope at this point. 

In describing the functionality of our robot, we have purposefully not specified the exact type of writing implement. There are many possible tools including by not limited to chalk, \deleted[remark={Confusing since assumptions discount this.}]{spray paint,} marker, liquid chalk, etc. While we hope to explore several of these options, we do not anticipate being able to explore all options equally. 

Delving further into writing implements there are also many smaller, interesting sub-points that we do not believe we will be able to build, such as drawing with multiple colors simultaneously or accounting for a large variety of sized writing tools. 
\rhnote{Need to be more clear about the scope of the item we are drawing.}

\subsection{Assumptions}
\label{sec:assumptions}

In designing and parameterizing the needs of our system we will make the following assumptions about or environment and operation: 
\begin{list}{A\arabic{qcounter}:~}{\usecounter{qcounter}}
\item \added[remark={Clarifying Assumption}]{We assume that our drawing surface is free of any obstacles.} 
\item \added[remark={Clarifying Assumption}]{We assume that our system functions indoors, thus removing the need to handle various weather conditions.}  
\item We assume that the robots are working on flat, homogenous surface. This disqualifies uneven or muddy ground, which is considered out of scope (\sref{sec:project_scope}).
\item We assume that the writing implement being used by each robot in the system can be loaded into the robot manually by a human. Thus we do not expect our robots to auto-load writing tools. 
\item We assume that the writing implement can by used by making contact between the tip of the implement and ground, such as a pencil. This assumption removes using writing tools like spray paint.
\deleted[remark={Moving assumption to requirement}] {We assume that between the robots and any controlling host we have near perfect, clear communication. Therefore we will not account for scenarios with excessive noise that would compromise robot communication.}
\end{list}